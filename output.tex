% Options for packages loaded elsewhere
\PassOptionsToPackage{unicode}{hyperref}
\PassOptionsToPackage{hyphens}{url}
%
\documentclass[
]{article}
\usepackage{amsmath,amssymb}
\usepackage{iftex}
\ifPDFTeX
  \usepackage[T1]{fontenc}
  \usepackage[utf8]{inputenc}
  \usepackage{textcomp} % provide euro and other symbols
\else % if luatex or xetex
  \usepackage{unicode-math} % this also loads fontspec
  \defaultfontfeatures{Scale=MatchLowercase}
  \defaultfontfeatures[\rmfamily]{Ligatures=TeX,Scale=1}
\fi
\usepackage{lmodern}
\ifPDFTeX\else
  % xetex/luatex font selection
\fi
% Use upquote if available, for straight quotes in verbatim environments
\IfFileExists{upquote.sty}{\usepackage{upquote}}{}
\IfFileExists{microtype.sty}{% use microtype if available
  \usepackage[]{microtype}
  \UseMicrotypeSet[protrusion]{basicmath} % disable protrusion for tt fonts
}{}
\makeatletter
\@ifundefined{KOMAClassName}{% if non-KOMA class
  \IfFileExists{parskip.sty}{%
    \usepackage{parskip}
  }{% else
    \setlength{\parindent}{0pt}
    \setlength{\parskip}{6pt plus 2pt minus 1pt}}
}{% if KOMA class
  \KOMAoptions{parskip=half}}
\makeatother
\usepackage{xcolor}
\usepackage{longtable,booktabs,array}
\usepackage{multirow}
\usepackage{calc} % for calculating minipage widths
% Correct order of tables after \paragraph or \subparagraph
\usepackage{etoolbox}
\makeatletter
\patchcmd\longtable{\par}{\if@noskipsec\mbox{}\fi\par}{}{}
\makeatother
% Allow footnotes in longtable head/foot
\IfFileExists{footnotehyper.sty}{\usepackage{footnotehyper}}{\usepackage{footnote}}
\makesavenoteenv{longtable}
\usepackage{graphicx}
\makeatletter
\def\maxwidth{\ifdim\Gin@nat@width>\linewidth\linewidth\else\Gin@nat@width\fi}
\def\maxheight{\ifdim\Gin@nat@height>\textheight\textheight\else\Gin@nat@height\fi}
\makeatother
% Scale images if necessary, so that they will not overflow the page
% margins by default, and it is still possible to overwrite the defaults
% using explicit options in \includegraphics[width, height, ...]{}
\setkeys{Gin}{width=\maxwidth,height=\maxheight,keepaspectratio}
% Set default figure placement to htbp
\makeatletter
\def\fps@figure{htbp}
\makeatother
\usepackage{svg}
\setlength{\emergencystretch}{3em} % prevent overfull lines
\providecommand{\tightlist}{%
  \setlength{\itemsep}{0pt}\setlength{\parskip}{0pt}}
\setcounter{secnumdepth}{-\maxdimen} % remove section numbering
\ifLuaTeX
\usepackage[bidi=basic]{babel}
\else
\usepackage[bidi=default]{babel}
\fi
\babelprovide[main,import]{english}
\babelprovide[import]{english}
% get rid of language-specific shorthands (see #6817):
\let\LanguageShortHands\languageshorthands
\def\languageshorthands#1{}
\ifLuaTeX
  \usepackage{selnolig}  % disable illegal ligatures
\fi
\ifPDFTeX
  \TeXXeTstate=1
  \newcommand{\RL}[1]{\beginR #1\endR}
  \newcommand{\LR}[1]{\beginL #1\endL}
  \newenvironment{RTL}{\beginR}{\endR}
  \newenvironment{LTR}{\beginL}{\endL}
\fi
\IfFileExists{bookmark.sty}{\usepackage{bookmark}}{\usepackage{hyperref}}
\IfFileExists{xurl.sty}{\usepackage{xurl}}{} % add URL line breaks if available
\urlstyle{same}
\hypersetup{
  pdftitle={Joule - Wikipedia},
  pdflang={en},
  hidelinks,
  pdfcreator={LaTeX via pandoc}}

\title{Joule - Wikipedia}
\author{}
\date{}

\begin{document}
\maketitle

\hyperref[bodyContent]{Jump to content}

\phantomsection\label{vector-main-menu-dropdown}
{} {Main menu}

\phantomsection\label{vector-main-menu-unpinned-container}
\phantomsection\label{vector-main-menu}
Main menu

move to sidebar

hide

\phantomsection\label{p-navigation}
Navigation

\begin{itemize}
\tightlist
\item
  \phantomsection\label{n-mainpage-description}{\href{/wiki/Main_Page}{{Main
  page}}}
\item
  \phantomsection\label{n-contents}{\href{/wiki/Wikipedia:Contents}{{Contents}}}
\item
  \phantomsection\label{n-currentevents}{\href{/wiki/Portal:Current_events}{{Current
  events}}}
\item
  \phantomsection\label{n-randompage}{\href{/wiki/Special:Random}{{Random
  article}}}
\item
  \phantomsection\label{n-aboutsite}{\href{/wiki/Wikipedia:About}{{About
  Wikipedia}}}
\item
  \phantomsection\label{n-contactpage}{\href{//en.wikipedia.org/wiki/Wikipedia:Contact_us}{{Contact
  us}}}
\item
  \phantomsection\label{n-sitesupport}{\href{https://donate.wikimedia.org/wiki/Special:FundraiserRedirector?utm_source=donate&utm_medium=sidebar&utm_campaign=C13_en.wikipedia.org&uselang=en}{{Donate}}}
\end{itemize}

\phantomsection\label{p-interaction}
Contribute

\begin{itemize}
\tightlist
\item
  \phantomsection\label{n-help}{\href{/wiki/Help:Contents}{{Help}}}
\item
  \phantomsection\label{n-introduction}{\href{/wiki/Help:Introduction}{{Learn
  to edit}}}
\item
  \phantomsection\label{n-portal}{\href{/wiki/Wikipedia:Community_portal}{{Community
  portal}}}
\item
  \phantomsection\label{n-recentchanges}{\href{/wiki/Special:RecentChanges}{{Recent
  changes}}}
\item
  \phantomsection\label{n-upload}{\href{/wiki/Wikipedia:File_upload_wizard}{{Upload
  file}}}
\end{itemize}

Languages

{}

Language links are at the top of the page across from the title.

\href{/wiki/Main_Page}{\includegraphics[width=0.52083in,height=0.52083in]{/static/images/icons/wikipedia.png}
{ \includesvg{/static/images/mobile/copyright/wikipedia-wordmark-en.svg}
\includesvg[width=1.21875in,height=0.13542in]{/static/images/mobile/copyright/wikipedia-tagline-en.svg}
}}

\phantomsection\label{p-search}
\href{/wiki/Special:Search}{{} {Search}}

\phantomsection\label{simpleSearch}
{}

Search

\phantomsection\label{p-vector-user-menu-preferences}

\phantomsection\label{p-vector-user-menu-userpage}

\phantomsection\label{p-vector-user-menu-notifications}

\phantomsection\label{p-vector-user-menu-overflow}
\begin{itemize}
\tightlist
\item
  \phantomsection\label{pt-createaccount-2}{\href{/w/index.php?title=Special:CreateAccount&returnto=Joule}{{Create
  account}}}
\item
  \phantomsection\label{pt-login-2}{\href{/w/index.php?title=Special:UserLogin&returnto=Joule}{{Log
  in}}}
\end{itemize}

\phantomsection\label{vector-user-links-dropdown}
{} {Personal tools}

\phantomsection\label{p-personal}
\begin{itemize}
\tightlist
\item
  \phantomsection\label{pt-createaccount}{\href{/w/index.php?title=Special:CreateAccount&returnto=Joule}{{}
  {Create account}}}
\item
  \phantomsection\label{pt-login}{\href{/w/index.php?title=Special:UserLogin&returnto=Joule}{{}
  {Log in}}}
\end{itemize}

\phantomsection\label{p-user-menu-anon-editor}
Pages for logged out editors \href{/wiki/Help:Introduction}{{learn
more}}

\begin{itemize}
\tightlist
\item
  \phantomsection\label{pt-anoncontribs}{\href{/wiki/Special:MyContributions}{{Contributions}}}
\item
  \phantomsection\label{pt-anontalk}{\href{/wiki/Special:MyTalk}{{Talk}}}
\end{itemize}

\phantomsection\label{siteNotice}

\phantomsection\label{mw-navigation}
\phantomsection\label{vector-main-menu-pinned-container}

\phantomsection\label{vector-toc-pinned-container}
\phantomsection\label{vector-toc}
\subsection{Contents}\label{contents}

move to sidebar

hide

\begin{itemize}
\tightlist
\item
  \phantomsection\label{toc-mw-content-text}{\hyperref[]{}}

  (Top)
\item
  \phantomsection\label{toc-Definition}{\hyperref[Definition]{}}

  {1}Definition
\item
  \phantomsection\label{toc-History}{\hyperref[History]{}}

  {2}History
\item
  \phantomsection\label{toc-Practical_examples}{\hyperref[Practical_examples]{}}

  {3}Practical examples
\item
  \phantomsection\label{toc-Multiples}{\hyperref[Multiples]{}}

  {4}Multiples
\item
  \phantomsection\label{toc-Conversions}{\hyperref[Conversions]{}}

  {5}Conversions
\item
  \phantomsection\label{toc-Newton-metre_and_torque}{\hyperref[Newton-metre_and_torque]{}}

  {6}Newton-metre and torque
\item
  \phantomsection\label{toc-Watt-second}{\hyperref[Watt-second]{}}

  {7}Watt-second
\item
  \phantomsection\label{toc-Notes}{\hyperref[Notes]{}}

  {8}Notes
\item
  \phantomsection\label{toc-References}{\hyperref[References]{}}

  {9}References
\item
  \phantomsection\label{toc-External_links}{\hyperref[External_links]{}}

  {10}External links
\end{itemize}

\phantomsection\label{content}
\phantomsection\label{vector-page-titlebar-toc}
{} {Toggle the table of contents}

\phantomsection\label{vector-page-titlebar-toc-unpinned-container}

\section{\texorpdfstring{{Joule}}{Joule}}\label{firstHeading}

\phantomsection\label{p-lang-btn}
{} {107 languages}

\begin{itemize}
\tightlist
\item
  \href{https://af.wikipedia.org/wiki/Joule}{{Afrikaans}}
\item
  \href{https://als.wikipedia.org/wiki/Joule}{{Alemannisch}}
\item
  \href{https://ar.wikipedia.org/wiki/\%D8\%AC\%D9\%88\%D9\%84}{{العربية}}
\item
  \href{https://an.wikipedia.org/wiki/Joule}{{Aragonés}}
\item
  \href{https://as.wikipedia.org/wiki/\%E0\%A6\%9C\%E0\%A7\%81\%E0\%A6\%B2}{{অসমীয়া}}
\item
  \href{https://ast.wikipedia.org/wiki/Xuliu_(unid\%C3\%A1)}{{Asturianu}}
\item
  \href{https://az.wikipedia.org/wiki/Coul}{{Azərbaycanca}}
\item
  \href{https://azb.wikipedia.org/wiki/\%DA\%98\%D9\%88\%D9\%84}{{تۆرکجه}}
\item
  \href{https://bn.wikipedia.org/wiki/\%E0\%A6\%9C\%E0\%A7\%81\%E0\%A6\%B2}{{বাংলা}}
\item
  \href{https://be.wikipedia.org/wiki/\%D0\%94\%D0\%B6\%D0\%BE\%D1\%9E\%D0\%BB\%D1\%8C_(\%D0\%B0\%D0\%B4\%D0\%B7\%D1\%96\%D0\%BD\%D0\%BA\%D0\%B0_\%D0\%B2\%D1\%8B\%D0\%BC\%D1\%8F\%D1\%80\%D1\%8D\%D0\%BD\%D0\%BD\%D1\%8F)}{{Беларуская}}
\item
  \href{https://be-tarask.wikipedia.org/wiki/\%D0\%94\%D0\%B6\%D0\%BE\%D1\%9E\%D0\%BB\%D1\%8C}{{Беларуская
  (тарашкевіца)}}
\item
  \href{https://bg.wikipedia.org/wiki/\%D0\%94\%D0\%B6\%D0\%B0\%D1\%83\%D0\%BB}{{Български}}
\item
  \href{https://bo.wikipedia.org/wiki/\%E0\%BD\%85\%E0\%BD\%BC\%E0\%BD\%A3\%E0\%BC\%8D}{{བོད་ཡིག}}
\item
  \href{https://bs.wikipedia.org/wiki/D\%C5\%BEul}{{Bosanski}}
\item
  \href{https://br.wikipedia.org/wiki/Joul}{{Brezhoneg}}
\item
  \href{https://ca.wikipedia.org/wiki/Joule}{{Català}}
\item
  \href{https://cv.wikipedia.org/wiki/\%D0\%94\%D0\%B6\%D0\%BE\%D1\%83\%D0\%BB\%D1\%8C}{{Чӑвашла}}
\item
  \href{https://cs.wikipedia.org/wiki/Joule}{{Čeština}}
\item
  \href{https://cy.wikipedia.org/wiki/Joule}{{Cymraeg}}
\item
  \href{https://da.wikipedia.org/wiki/Joule}{{Dansk}}
\item
  \href{https://de.wikipedia.org/wiki/Joule}{{Deutsch}}
\item
  \href{https://et.wikipedia.org/wiki/D\%C5\%BEaul}{{Eesti}}
\item
  \href{https://el.wikipedia.org/wiki/\%CE\%A4\%CE\%B6\%CE\%AC\%CE\%BF\%CF\%85\%CE\%BB_(\%CE\%BC\%CE\%BF\%CE\%BD\%CE\%AC\%CE\%B4\%CE\%B1_\%CE\%BC\%CE\%AD\%CF\%84\%CF\%81\%CE\%B7\%CF\%83\%CE\%B7\%CF\%82)}{{Ελληνικά}}
\item
  \href{https://es.wikipedia.org/wiki/Julio_(unidad)}{{Español}}
\item
  \href{https://eo.wikipedia.org/wiki/\%C4\%B4ulo}{{Esperanto}}
\item
  \href{https://eu.wikipedia.org/wiki/Joule_(unitatea)}{{Euskara}}
\item
  \href{https://fa.wikipedia.org/wiki/\%DA\%98\%D9\%88\%D9\%84}{{فارسی}}
\item
  \href{https://fr.wikipedia.org/wiki/Joule}{{Français}}
\item
  \href{https://ga.wikipedia.org/wiki/Gi\%C3\%BAl}{{Gaeilge}}
\item
  \href{https://gd.wikipedia.org/wiki/Joule}{{Gàidhlig}}
\item
  \href{https://gl.wikipedia.org/wiki/Joule}{{Galego}}
\item
  \href{https://gu.wikipedia.org/wiki/\%E0\%AA\%9C\%E0\%AB\%82\%E0\%AA\%B2}{{ગુજરાતી}}
\item
  \href{https://ko.wikipedia.org/wiki/\%EC\%A4\%84_(\%EB\%8B\%A8\%EC\%9C\%84)}{{한국어}}
\item
  \href{https://hy.wikipedia.org/wiki/\%D5\%8B\%D5\%B8\%D5\%B8\%D6\%82\%D5\%AC}{{Հայերեն}}
\item
  \href{https://hi.wikipedia.org/wiki/\%E0\%A4\%9C\%E0\%A5\%82\%E0\%A4\%B2_(\%E0\%A4\%87\%E0\%A4\%95\%E0\%A4\%BE\%E0\%A4\%88)}{{हिन्दी}}
\item
  \href{https://hr.wikipedia.org/wiki/D\%C5\%BEul}{{Hrvatski}}
\item
  \href{https://id.wikipedia.org/wiki/Joule}{{Bahasa Indonesia}}
\item
  \href{https://ia.wikipedia.org/wiki/Joule}{{Interlingua}}
\item
  \href{https://is.wikipedia.org/wiki/J\%C3\%BAl}{{Íslenska}}
\item
  \href{https://it.wikipedia.org/wiki/Joule}{{Italiano}}
\item
  \href{https://he.wikipedia.org/wiki/\%D7\%92\%27\%D7\%95\%D7\%9C}{{עברית}}
\item
  \href{https://krc.wikipedia.org/wiki/\%D0\%94\%D0\%B6\%D0\%BE\%D1\%83\%D0\%BB\%D1\%8C}{{Къарачай-малкъар}}
\item
  \href{https://ka.wikipedia.org/wiki/\%E1\%83\%AF\%E1\%83\%9D\%E1\%83\%A3\%E1\%83\%9A\%E1\%83\%98}{{ქართული}}
\item
  \href{https://kk.wikipedia.org/wiki/\%D0\%94\%D0\%B6\%D0\%BE\%D1\%83\%D0\%BB\%D1\%8C}{{Қазақша}}
\item
  \href{https://kw.wikipedia.org/wiki/Joule}{{Kernowek}}
\item
  \href{https://sw.wikipedia.org/wiki/Jouli}{{Kiswahili}}
\item
  \href{https://ht.wikipedia.org/wiki/Joul}{{Kreyòl ayisyen}}
\item
  \href{https://ku.wikipedia.org/wiki/Joule}{{Kurdî}}
\item
  \href{https://ky.wikipedia.org/wiki/\%D0\%92\%D0\%B0\%D1\%82\%D1\%82-\%D1\%81\%D0\%B5\%D0\%BA\%D1\%83\%D0\%BD\%D0\%B4\%D0\%B0}{{Кыргызча}}
\item
  \href{https://la.wikipedia.org/wiki/Joulium}{{Latina}}
\item
  \href{https://lv.wikipedia.org/wiki/D\%C5\%BEouls}{{Latviešu}}
\item
  \href{https://lt.wikipedia.org/wiki/D\%C5\%BEaulis}{{Lietuvių}}
\item
  \href{https://li.wikipedia.org/wiki/Joule}{{Limburgs}}
\item
  \href{https://lmo.wikipedia.org/wiki/Joule}{{Lombard}}
\item
  \href{https://hu.wikipedia.org/wiki/Joule}{{Magyar}}
\item
  \href{https://mk.wikipedia.org/wiki/\%D0\%8F\%D1\%83\%D0\%BB}{{Македонски}}
\item
  \href{https://ml.wikipedia.org/wiki/\%E0\%B4\%9C\%E0\%B5\%82\%E0\%B5\%BE}{{മലയാളം}}
\item
  \href{https://mr.wikipedia.org/wiki/\%E0\%A4\%9C\%E0\%A5\%8D\%E0\%A4\%AF\%E0\%A5\%82\%E0\%A4\%B2}{{मराठी}}
\item
  \href{https://arz.wikipedia.org/wiki/\%D8\%AC\%D9\%88\%D9\%84_(\%D8\%B7\%D8\%A7\%D9\%82\%D8\%A9)}{{مصرى}}
\item
  \href{https://ms.wikipedia.org/wiki/Joule}{{Bahasa Melayu}}
\item
  \href{https://mn.wikipedia.org/wiki/\%D0\%96\%D0\%BE\%D1\%83\%D0\%BB\%D1\%8C}{{Монгол}}
\item
  \href{https://my.wikipedia.org/wiki/\%E1\%80\%82\%E1\%80\%BB\%E1\%80\%AD\%E1\%80\%AF\%E1\%80\%B8}{{မြန်မာဘာသာ}}
\item
  \href{https://nl.wikipedia.org/wiki/Joule}{{Nederlands}}
\item
  \href{https://ja.wikipedia.org/wiki/\%E3\%82\%B8\%E3\%83\%A5\%E3\%83\%BC\%E3\%83\%AB}{{日本語}}
\item
  \href{https://frr.wikipedia.org/wiki/Joule_(ianhaid)}{{Nordfriisk}}
\item
  \href{https://no.wikipedia.org/wiki/Joule}{{Norsk bokmål}}
\item
  \href{https://nn.wikipedia.org/wiki/Joule}{{Norsk nynorsk}}
\item
  \href{https://oc.wikipedia.org/wiki/Joule}{{Occitan}}
\item
  \href{https://uz.wikipedia.org/wiki/Joul}{{Oʻzbekcha / ўзбекча}}
\item
  \href{https://pa.wikipedia.org/wiki/\%E0\%A8\%9C\%E0\%A9\%82\%E0\%A8\%B2}{{ਪੰਜਾਬੀ}}
\item
  \href{https://pnb.wikipedia.org/wiki/\%D8\%AC\%D8\%A7\%D8\%A4\%D9\%84}{{پنجابی}}
\item
  \href{https://pms.wikipedia.org/wiki/Joule}{{Piemontèis}}
\item
  \href{https://pl.wikipedia.org/wiki/D\%C5\%BCul}{{Polski}}
\item
  \href{https://pt.wikipedia.org/wiki/Joule}{{Português}}
\item
  \href{https://ro.wikipedia.org/wiki/Joule}{{Română}}
\item
  \href{https://ru.wikipedia.org/wiki/\%D0\%94\%D0\%B6\%D0\%BE\%D1\%83\%D0\%BB\%D1\%8C}{{Русский}}
\item
  \href{https://sc.wikipedia.org/wiki/Joule}{{Sardu}}
\item
  \href{https://sco.wikipedia.org/wiki/Joule}{{Scots}}
\item
  \href{https://sq.wikipedia.org/wiki/Xhaul}{{Shqip}}
\item
  \href{https://scn.wikipedia.org/wiki/Joule}{{Sicilianu}}
\item
  \href{https://si.wikipedia.org/wiki/\%E0\%B6\%A2\%E0\%B7\%96\%E0\%B6\%BD\%E0\%B7\%8A_(SI_\%E0\%B6\%92\%E0\%B6\%9A\%E0\%B6\%9A\%E0\%B6\%BA)}{{සිංහල}}
\item
  \href{https://simple.wikipedia.org/wiki/Joule}{{Simple English}}
\item
  \href{https://sk.wikipedia.org/wiki/Joule}{{Slovenčina}}
\item
  \href{https://sl.wikipedia.org/wiki/D\%C5\%BEul}{{Slovenščina}}
\item
  \href{https://szl.wikipedia.org/wiki/D\%C5\%BCul}{{Ślůnski}}
\item
  \href{https://ckb.wikipedia.org/wiki/\%D8\%AC\%D9\%88\%D9\%88\%DA\%B5}{{کوردی}}
\item
  \href{https://sr.wikipedia.org/wiki/\%D0\%8F\%D1\%83\%D0\%BB}{{Српски
  / srpski}}
\item
  \href{https://sh.wikipedia.org/wiki/D\%C5\%BEul}{{Srpskohrvatski /
  српскохрватски}}
\item
  \href{https://su.wikipedia.org/wiki/Joule}{{Sunda}}
\item
  \href{https://fi.wikipedia.org/wiki/Joule}{{Suomi}}
\item
  \href{https://sv.wikipedia.org/wiki/Joule}{{Svenska}}
\item
  \href{https://ta.wikipedia.org/wiki/\%E0\%AE\%9A\%E0\%AF\%82\%E0\%AE\%B2\%E0\%AF\%8D_(\%E0\%AE\%85\%E0\%AE\%B2\%E0\%AE\%95\%E0\%AF\%81)}{{தமிழ்}}
\item
  \href{https://roa-tara.wikipedia.org/wiki/Joule}{{Tarandíne}}
\item
  \href{https://tt.wikipedia.org/wiki/\%D0\%94\%D0\%B6\%D0\%BE\%D1\%83\%D0\%BB\%D1\%8C}{{Татарча
  / tatarça}}
\item
  \href{https://te.wikipedia.org/wiki/\%E0\%B0\%9C\%E0\%B1\%8C\%E0\%B0\%B2\%E0\%B1\%8D}{{తెలుగు}}
\item
  \href{https://th.wikipedia.org/wiki/\%E0\%B8\%88\%E0\%B8\%B9\%E0\%B8\%A5_(\%E0\%B8\%AB\%E0\%B8\%99\%E0\%B9\%88\%E0\%B8\%A7\%E0\%B8\%A2)}{{ไทย}}
\item
  \href{https://tr.wikipedia.org/wiki/Joule}{{Türkçe}}
\item
  \href{https://uk.wikipedia.org/wiki/\%D0\%94\%D0\%B6\%D0\%BE\%D1\%83\%D0\%BB\%D1\%8C}{{Українська}}
\item
  \href{https://ur.wikipedia.org/wiki/\%D8\%AC\%D9\%88\%D9\%84}{{اردو}}
\item
  \href{https://vec.wikipedia.org/wiki/Joule}{{Vèneto}}
\item
  \href{https://vi.wikipedia.org/wiki/Joule}{{Tiếng Việt}}
\item
  \href{https://zh-classical.wikipedia.org/wiki/\%E7\%84\%A6\%E8\%80\%B3}{{文言}}
\item
  \href{https://war.wikipedia.org/wiki/Joule}{{Winaray}}
\item
  \href{https://wuu.wikipedia.org/wiki/\%E7\%84\%A6\%E8\%80\%B3}{{吴语}}
\item
  \href{https://zh-yue.wikipedia.org/wiki/\%E7\%84\%A6\%E8\%80\%B3}{{粵語}}
\item
  \href{https://bat-smg.wikipedia.org/wiki/D\%C5\%BEaul\%C4\%97s}{{Žemaitėška}}
\item
  \href{https://zh.wikipedia.org/wiki/\%E7\%84\%A6\%E8\%80\%B3}{{中文}}
\end{itemize}

{\href{https://www.wikidata.org/wiki/Special:EntityPage/Q25269\#sitelinks-wikipedia}{Edit
links}}

\phantomsection\label{left-navigation}
\phantomsection\label{p-associated-pages}
\begin{itemize}
\tightlist
\item
  \phantomsection\label{ca-nstab-main}{\href{/wiki/Joule}{{Article}}}
\item
  \phantomsection\label{ca-talk}{\href{/wiki/Talk:Joule}{{Talk}}}
\end{itemize}

\phantomsection\label{p-variants}
{English}

\phantomsection\label{p-variants}

\phantomsection\label{right-navigation}
\phantomsection\label{p-views}
\begin{itemize}
\tightlist
\item
  \phantomsection\label{ca-view}{\href{/wiki/Joule}{{Read}}}
\item
  \phantomsection\label{ca-edit}{\href{/w/index.php?title=Joule&action=edit}{{Edit}}}
\item
  \phantomsection\label{ca-history}{\href{/w/index.php?title=Joule&action=history}{{View
  history}}}
\end{itemize}

\phantomsection\label{vector-page-tools-dropdown}
{Tools}

\phantomsection\label{vector-page-tools-unpinned-container}
\phantomsection\label{vector-page-tools}
Tools

move to sidebar

hide

\phantomsection\label{p-cactions}
Actions

\begin{itemize}
\tightlist
\item
  \phantomsection\label{ca-more-view}{\href{/wiki/Joule}{{Read}}}
\item
  \phantomsection\label{ca-more-edit}{\href{/w/index.php?title=Joule&action=edit}{{Edit}}}
\item
  \phantomsection\label{ca-more-history}{\href{/w/index.php?title=Joule&action=history}{{View
  history}}}
\end{itemize}

\phantomsection\label{p-tb}
General

\begin{itemize}
\tightlist
\item
  \phantomsection\label{t-whatlinkshere}{\href{/wiki/Special:WhatLinksHere/Joule}{{What
  links here}}}
\item
  \phantomsection\label{t-recentchangeslinked}{\href{/wiki/Special:RecentChangesLinked/Joule}{{Related
  changes}}}
\item
  \phantomsection\label{t-upload}{\href{/wiki/Wikipedia:File_Upload_Wizard}{{Upload
  file}}}
\item
  \phantomsection\label{t-specialpages}{\href{/wiki/Special:SpecialPages}{{Special
  pages}}}
\item
  \phantomsection\label{t-permalink}{\href{/w/index.php?title=Joule&oldid=1187127150}{{Permanent
  link}}}
\item
  \phantomsection\label{t-info}{\href{/w/index.php?title=Joule&action=info}{{Page
  information}}}
\item
  \phantomsection\label{t-cite}{\href{/w/index.php?title=Special:CiteThisPage&page=Joule&id=1187127150&wpFormIdentifier=titleform}{{Cite
  this page}}}
\item
  \phantomsection\label{t-urlshortener}{\href{/w/index.php?title=Special:UrlShortener&url=https\%3A\%2F\%2Fen.wikipedia.org\%2Fwiki\%2FJoule}{{Get
  shortened URL}}}
\item
  \phantomsection\label{t-wikibase}{\href{https://www.wikidata.org/wiki/Special:EntityPage/Q25269}{{Wikidata
  item}}}
\end{itemize}

\phantomsection\label{p-coll-print_export}
Print/export

\begin{itemize}
\tightlist
\item
  \phantomsection\label{coll-download-as-rl}{\href{/w/index.php?title=Special:DownloadAsPdf&page=Joule&action=show-download-screen}{{Download
  as PDF}}}
\item
  \phantomsection\label{t-print}{\href{/w/index.php?title=Joule&printable=yes}{{Printable
  version}}}
\end{itemize}

\phantomsection\label{p-wikibase-otherprojects}
In other projects

\begin{itemize}
\tightlist
\item
  \href{https://commons.wikimedia.org/wiki/Category:Joule_(unit)}{{Wikimedia
  Commons}}
\end{itemize}

\phantomsection\label{vector-page-tools-pinned-container}

\phantomsection\label{bodyContent}
\phantomsection\label{siteSub}
From Wikipedia, the free encyclopedia

\phantomsection\label{contentSub}
\phantomsection\label{mw-content-subtitle}

\phantomsection\label{mw-content-text}
\begin{LTR}
\begin{otherlanguage}{english}

Unit of energy in the SI system

This article is about the unit of energy or work. For other uses, see
\href{/wiki/Joule_(disambiguation)}{Joule (disambiguation)}.

\begin{longtable}[]{@{}ll@{}}
\toprule\noalign{}
\endhead
\bottomrule\noalign{}
\endlastfoot
\multicolumn{2}{@{}l@{}}{%
joule} \\
\href{/wiki/System_of_measurement}{Unit system} & \href{/wiki/SI}{SI} \\
Unit~of & \href{/wiki/Energy}{energy} \\
Symbol & J \\
Named after & \href{/wiki/James_Prescott_Joule}{James Prescott Joule} \\
\multicolumn{2}{@{}l@{}}{%
Conversions} \\
1~J \emph{in ...} & \emph{... is equal to ...} \\
\multicolumn{2}{@{}l@{}}{%
} \\
{~~~}\href{/wiki/SI_base_unit}{SI base units} &
{~~~}\href{/wiki/Kilogram}{kg}⋅\href{/wiki/Metre}{m}\textsuperscript{2}⋅\href{/wiki/Second}{s}\textsuperscript{−2} \\
{~~~}\href{/wiki/CGS_unit}{CGS units} &
{~~~}{{}1{×}10\textsuperscript{7}} \href{/wiki/Erg}{erg} \\
{~~~}\hyperref[Watt-second]{watt-seconds} &
{~~~}{{}1}~\href{/wiki/Watt}{W}⋅\href{/wiki/Second}{s} \\
{~~~}\href{/wiki/Kilowatt-hour}{kilowatt-hours} &
{~~~}≈{{}2.78{×}10\textsuperscript{−7}~kW⋅h} \\
{~~~}\href{/wiki/Kilocalorie}{kilocalories}
(\href{/wiki/Thermochemistry}{thermochemical}) &
{~~~}{{}2.390{×}10\textsuperscript{−4}~kcal\textsubscript{th}} \\
{~~~}\href{/wiki/British_thermal_unit}{BTUs} &
{~~~}{{}9.48{×}10\textsuperscript{−4}~BTU} \\
{~~~}\href{/wiki/Electronvolt}{electronvolts} &
{~~~}≈{{}6.24{×}10\textsuperscript{18}~eV} \\
\end{longtable}

The \textbf{joule} (pronounced
{\foreignlanguage{english}{\href{/wiki/Help:IPA/English}{/{{ˈ}{dʒ}{uː}{l}}/}}},
\href{/wiki/Help:Pronunciation_respelling_key}{\emph{{JOOL}}} or
{\foreignlanguage{english}{\href{/wiki/Help:IPA/English}{/{{ˈ}{dʒ}{aʊ}{l}}/}}}
\href{/wiki/Help:Pronunciation_respelling_key}{\emph{{JOWL}}}; symbol:
\textbf{J}) is the unit of \href{/wiki/Energy}{energy} in the
\href{/wiki/International_System_of_Units}{International System of Units
(SI)}.\textsuperscript{\hyperref[cite_note-1]{{[}1{]}}} It is equal to
the amount of \href{/wiki/Work_(physics)}{work} done when a force of 1
\href{/wiki/Newton_(unit)}{newton} displaces a mass through a distance
of 1 \href{/wiki/Metre}{metre} in the direction of the force applied. It
is also the energy dissipated as heat when an
\href{/wiki/Electric_current}{electric current} of one
\href{/wiki/Ampere}{ampere} passes through a
\href{/wiki/Electrical_resistance_and_conductance}{resistance} of one
\href{/wiki/Ohm}{ohm} for one second. It is named after the English
physicist \href{/wiki/James_Prescott_Joule}{James Prescott Joule}
(1818--1889).\textsuperscript{\hyperref[cite_note-2]{{[}2{]}}\hyperref[cite_note-3]{{[}3{]}}\hyperref[cite_note-4]{{[}4{]}}}

\subsection[{{{[}}\href{/w/index.php?title=Joule&action=edit&section=1}{{edit}}{{]}}}]{\texorpdfstring{\phantomsection\label{Definition}{Definition}{{{[}}\href{/w/index.php?title=Joule&action=edit&section=1}{{edit}}{{]}}}}{Definition{[}edit{]}}}\label{definitionedit}

In terms of \href{/wiki/SI_base_units}{SI base units} and in terms of
\href{/wiki/SI_derived_units_with_special_names}{SI derived units with
special names}, the joule is defined as

\begin{longtable}[]{@{}
  >{\raggedright\arraybackslash}p{(\columnwidth - 2\tabcolsep) * \real{0.5000}}
  >{\raggedright\arraybackslash}p{(\columnwidth - 2\tabcolsep) * \real{0.5000}}@{}}
\toprule\noalign{}
\endhead
\bottomrule\noalign{}
\endlastfoot
\begin{minipage}[t]{\linewidth}\raggedright
\[\begin{matrix}
{J\;} & {= \ {kg \cdot m^{2} \cdot s^{- 2}}} \\
 & {= \ {N \cdot m}} \\
 & {= \ {Pa \cdot m^{3}}} \\
 & {= \ {W \cdot s}} \\
 & {= \ {C \cdot V}}
\end{matrix}\]

\includegraphics{https://wikimedia.org/api/rest_v1/media/math/render/svg/9d8269cf201c3e42c686c724928b517b8ce69885}
\end{minipage} & \begin{minipage}[t]{\linewidth}\raggedright
\begin{longtable}[]{@{}ll@{}}
\toprule\noalign{}
\endhead
\bottomrule\noalign{}
\endlastfoot
Symbol & Meaning \\
J & joule \\
kg & \href{/wiki/Kilogram}{kilogram} \\
m & \href{/wiki/Metre}{metre} \\
s & \href{/wiki/Second}{second} \\
N & \href{/wiki/Newton_(unit)}{newton} \\
Pa & \href{/wiki/Pascal_(unit)}{pascal} \\
W & \href{/wiki/Watt}{watt} \\
C & \href{/wiki/Coulomb}{coulomb} \\
V & \href{/wiki/Volt}{volt} \\
\end{longtable}
\end{minipage} \\
\end{longtable}

One joule can also be defined by any of the following:

\begin{itemize}
\tightlist
\item
  The work required to move an \href{/wiki/Electric_charge}{electric
  charge} of one \href{/wiki/Coulomb}{coulomb} through an
  \href{/wiki/Voltage}{electrical potential difference} of one volt, or
  one coulomb-volt (C⋅V). This relationship can be used to define the
  volt.
\item
  The work required to produce one watt of
  \href{/wiki/Power_(physics)}{power} for one second, or one watt-second
  (W⋅s) (compare \href{/wiki/Kilowatt-hour}{kilowatt-hour}, which is 3.6
  megajoules). This relationship can be used to define the watt.
\end{itemize}

The joule is named after \href{/wiki/James_Prescott_Joule}{James
Prescott Joule}. As with every
\href{/wiki/International_System_of_Units}{SI} unit named for a person,
its symbol starts with an \href{/wiki/Upper_case}{upper case} letter
(J), but when written in full, it follows the rules for capitalisation
of a \href{/wiki/Common_noun}{common noun}; i.e., "\emph{joule}" becomes
capitalised at the beginning of a sentence and in titles but is
otherwise in lower case.

\subsection[{{{[}}\href{/w/index.php?title=Joule&action=edit&section=2}{{edit}}{{]}}}]{\texorpdfstring{\phantomsection\label{History}{History}{{{[}}\href{/w/index.php?title=Joule&action=edit&section=2}{{edit}}{{]}}}}{History{[}edit{]}}}\label{historyedit}

The \href{/wiki/Cgs_system}{cgs system} had been declared official in
1881, at the first
\href{/wiki/International_Electrical_Congress}{International Electrical
Congress}. The \href{/wiki/Erg}{erg} was adopted as its unit of energy
in 1882. \href{/wiki/Carl_Wilhelm_Siemens}{Wilhelm Siemens}, in his
inauguration speech as chairman of the
\href{/wiki/British_Association_for_the_Advancement_of_Science}{British
Association for the Advancement of Science} (23 August 1882) first
proposed the \emph{Joule} as unit of \href{/wiki/Heat}{heat}, to be
derived from the electromagnetic units \href{/wiki/Ampere}{Ampere} and
\href{/wiki/Ohm}{Ohm}, in cgs units equivalent to
{{}10\textsuperscript{7}~erg}. The naming of the unit in honour of
\href{/wiki/James_Prescott_Joule}{James Prescott Joule} (1818--1889), at
the time retired but still living (aged 63), is due to Siemens:

"Such a heat unit, if found acceptable, might with great propriety, I
think, be called the Joule, after the man who has done so much to
develop the dynamical theory of
heat."\textsuperscript{\hyperref[cite_note-5]{{[}5{]}}}

At the second International Electrical Congress, on 31 August 1889, the
joule was officially adopted alongside the \href{/wiki/Watt}{watt} and
the \emph{quadrant} (later renamed to
\href{/wiki/Henry_(unit)}{henry}).\textsuperscript{\hyperref[cite_note-6]{{[}6{]}}}
Joule died in the same year, on 11 October 1889. At the fourth congress
(1893), the "international ampere" and "international ohm" were defined,
with slight changes in the specifications for their measurement, with
the "international joule" being the unit derived from
them.\textsuperscript{\hyperref[cite_note-7]{{[}7{]}}}

In 1935, the
\href{/wiki/International_Electrotechnical_Commission}{International
Electrotechnical Commission} (as the successor organisation of the
International Electrical Congress) adopted the
"\href{/wiki/Giovanni_Giorgi}{Giorgi} system", which by virtue of
assuming a defined value for the \href{/wiki/Magnetic_constant}{magnetic
constant} also implied a redefinition of the Joule. The Giorgi system
was approved by the
\href{/wiki/International_Committee_for_Weights_and_Measures}{International
Committee for Weights and Measures} in 1946. The joule was now no longer
defined based on electromagnetic unit, but instead as the unit of
\href{/wiki/Work_(physics)}{work} performed by one unit of force (at the
time not yet named \href{/wiki/Newton_(unit)}{newton}) over the distance
of 1 \href{/wiki/Metre}{metre}. The joule was explicitly intended as the
unit of energy to be used in both electromagnetic and mechanical
contexts.\textsuperscript{\hyperref[cite_note-8]{{[}8{]}}} The
ratification of the definition at the ninth
\href{/wiki/General_Conference_on_Weights_and_Measures}{General
Conference on Weights and Measures}, in 1948, added the specification
that the joule was also to be preferred as the unit of
\href{/wiki/Heat}{heat} in the context of
\href{/wiki/Calorimetry}{calorimetry}, thereby officially deprecating
the use of the
\href{/wiki/Calorie}{calorie}.\textsuperscript{\hyperref[cite_note-9]{{[}9{]}}}
This definition was the direct precursor of the joule as adopted in the
modern \href{/wiki/International_System_of_Units}{International System
of Units} in 1960.

The definition of the joule as
J~=~kg⋅m\textsuperscript{2}⋅s\textsuperscript{−2} has remained unchanged
since 1946, but the joule as a derived unit has inherited changes in the
definitions of the \href{/wiki/Second}{second} (in 1960 and 1967), the
\href{/wiki/Metre}{metre} (in 1983) and the
\href{/wiki/Kilogram}{kilogram}
(\href{/wiki/2019_redefinition_of_the_SI_base_units}{in 2019}).

\subsection[{{{[}}\href{/w/index.php?title=Joule&action=edit&section=3}{{edit}}{{]}}}]{\texorpdfstring{\phantomsection\label{Practical_examples}{Practical
examples}{{{[}}\href{/w/index.php?title=Joule&action=edit&section=3}{{edit}}{{]}}}}{Practical examples{[}edit{]}}}\label{practical-examplesedit}

One joule represents (approximately):

\begin{itemize}
\tightlist
\item
  The amount of electricity required to run a {{}1~\href{/wiki/Watt}{W}}
  device for {{}1~\href{/wiki/Second}{s}}.
\item
  The energy required to accelerate a {{}1~\href{/wiki/Kilogram}{kg}}
  mass at
  {{}1~\href{/wiki/Metre_per_second_squared}{m/s\textsuperscript{2}}}
  through a distance of {{}1~\href{/wiki/Metre}{m}}.
\item
  The \href{/wiki/Kinetic_energy}{kinetic energy} of a
  {{}2~\href{/wiki/Kilogram}{kg}} \href{/wiki/Mass}{mass} travelling at
  {{}1~\href{/wiki/Metre_per_second}{m/s}}, or a
  {{}1~\href{/wiki/Kilogram}{kg}} mass travelling at
  {{}1.41~\href{/wiki/Metre_per_second}{m/s}}.
\item
  The energy required to lift an apple up 1 m, assuming the apple has a
  mass of 101.97 g.
\item
  The \href{/wiki/Heat}{heat} required to raise the temperature of 0.239
  g of water from 0 °C to 1
  °C.\textsuperscript{\hyperref[cite_note-10]{{[}10{]}}}
\item
  The typical energy released as heat by a person at rest every 1/60 s
  ({{}17~\href{/wiki/Millisecond}{ms}}).\textsuperscript{\hyperref[cite_note-11]{{[}note
  1{]}}}
\item
  The \href{/wiki/Kinetic_energy}{kinetic energy} of a {{}50~kg} human
  moving very slowly (0.2~m/s or 0.72~km/h).
\item
  The kinetic energy of a {{}56~g} tennis ball moving at 6~m/s
  (22~km/h).\textsuperscript{\hyperref[cite_note-12]{{[}11{]}}}
\item
  The food energy (kcal) in slightly more than half of an ordinary-sized
  sugar crystal ({{}0.102~\href{/wiki/Milligram}{mg}}/crystal).
\end{itemize}

\subsection[{{{[}}\href{/w/index.php?title=Joule&action=edit&section=4}{{edit}}{{]}}}]{\texorpdfstring{\phantomsection\label{Multiples}{Multiples}{{{[}}\href{/w/index.php?title=Joule&action=edit&section=4}{{edit}}{{]}}}}{Multiples{[}edit{]}}}\label{multiplesedit}

For additional examples, see
\href{/wiki/Orders_of_magnitude_(energy)}{Orders of magnitude (energy)}.

\begin{longtable}[]{@{}llllll@{}}
\caption{SI multiples of joule (J)}\tabularnewline
\toprule\noalign{}
\endfirsthead
\endhead
\bottomrule\noalign{}
\endlastfoot
\multicolumn{3}{@{}l}{%
Submultiples} & \multicolumn{3}{l@{}}{%
Multiples} \\
Value & SI symbol & Name & Value & SI symbol & Name \\
10\textsuperscript{−1}~J & dJ & decijoule & 10\textsuperscript{1}~J &
daJ & decajoule \\
10\textsuperscript{−2}~J & cJ & centijoule & 10\textsuperscript{2}~J &
hJ & hectojoule \\
10\textsuperscript{−3}~J & \textbf{mJ} & \textbf{millijoule} &
10\textsuperscript{3}~J & \textbf{kJ} & \textbf{kilojoule} \\
10\textsuperscript{−6}~J & \textbf{µJ} & \textbf{microjoule} &
10\textsuperscript{6}~J & \textbf{MJ} & \textbf{megajoule} \\
10\textsuperscript{−9}~J & \textbf{nJ} & \textbf{nanojoule} &
10\textsuperscript{9}~J & \textbf{GJ} & \textbf{gigajoule} \\
10\textsuperscript{−12}~J & \textbf{pJ} & \textbf{picojoule} &
10\textsuperscript{12}~J & \textbf{TJ} & \textbf{terajoule} \\
10\textsuperscript{−15}~J & fJ & femtojoule & 10\textsuperscript{15}~J &
PJ & petajoule \\
10\textsuperscript{−18}~J & aJ & attojoule & 10\textsuperscript{18}~J &
EJ & exajoule \\
10\textsuperscript{−21}~J & zJ & zeptojoule & 10\textsuperscript{21}~J &
ZJ & zettajoule \\
10\textsuperscript{−24}~J & yJ & yoctojoule & 10\textsuperscript{24}~J &
YJ & yottajoule \\
10\textsuperscript{−27}~J & rJ & rontojoule & 10\textsuperscript{27}~J &
RJ & ronnajoule \\
10\textsuperscript{−30}~J & qJ & quectojoule & 10\textsuperscript{30}~J
& QJ & quettajoule \\
\multicolumn{6}{@{}l@{}}{%
Common multiples are in bold face} \\
\end{longtable}

{\phantomsection\label{Zeptojoule}{}{Zeptojoule}}

{{}160~zeptojoule} is about one \href{/wiki/Electronvolt}{electronvolt}.

The minimal energy needed to change a bit of data in computation at
around room temperature -- approximately {{}2.75~zJ} -- is given by the
\href{/wiki/Landauer_limit}{Landauer limit}.

{\phantomsection\label{Nanojoule}{}{Nanojoule}}

{{}160~nanojoule} is about the \href{/wiki/Kinetic_energy}{kinetic
energy} of a flying
mosquito.\textsuperscript{\hyperref[cite_note-13]{{[}12{]}}}

{\phantomsection\label{Microjoule}{}{Microjoule}}

The \href{/wiki/Large_Hadron_Collider}{Large Hadron Collider} (LHC)
produces collisions of the microjoule order (7 TeV) per particle.

{\phantomsection\label{Kilojoule}{}{Kilojoule}}

Nutritional food labels in most countries express energy in kilojoules
(kJ).\textsuperscript{\hyperref[cite_note-Cal_vs_kJ-14]{{[}13{]}}}

One square metre of the \href{/wiki/Earth}{Earth} receives about
{{}1.4~kilojoules} of \href{/wiki/Sunlight\#Solar_constant}{solar
radiation} every second in full
daylight.\textsuperscript{\hyperref[cite_note-TSI-15]{{[}14{]}}} A human
in a sprint has approximately 3~kJ of kinetic
energy,\textsuperscript{\hyperref[cite_note-16]{{[}15{]}}} while a
cheetah in a {{}122~\href{/wiki/Kilometres_per_hour}{km/h}} (76~mph)
sprint has approximately
20~kJ.\textsuperscript{\hyperref[cite_note-17]{{[}16{]}}} One
\href{/wiki/Watt-hour}{watt-hour} of electricity is {{}3.6~kilojoules}.

{\phantomsection\label{Megajoule}{}{Megajoule}}

The megajoule is approximately the kinetic energy of a one megagram
(tonne) vehicle moving at {{}161~\href{/wiki/Kilometres_per_hour}{km/h}}
(100~mph).

The energy required to heat {{}10~L} of liquid water at constant
pressure from 0~°C (32~°F) to 100~°C (212~°F) is approximately
{{}4.2~\href{/wiki/Megajoule}{MJ}}.

One \href{/wiki/Kilowatt-hour}{kilowatt-hour} of electricity is
{{}3.6~megajoules}.

{\phantomsection\label{Gigajoule}{}{Gigajoule}}

{{}6~\href{/wiki/Gigajoule}{gigajoule}} is about the
\href{/wiki/Chemical_energy}{chemical energy} of combusting 1 barrel
(159~L) of
\href{/wiki/Petroleum}{petroleum}.\textsuperscript{\hyperref[cite_note-18]{{[}17{]}}}
2~GJ is about the \href{/wiki/Planck_units}{Planck energy} unit. One
\href{/wiki/Megawatt-hour}{megawatt-hour} of electricity is
{{}3.6~gigajoules}.

{\phantomsection\label{Terajoule}{}{Terajoule}}

The terajoule is about {{}0.278~\href{/wiki/Kilowatt-hour}{GWh}} (which
is often used in energy tables). About {{}63~\href{/wiki/Terajoule}{TJ}}
of energy was released by \href{/wiki/Little_Boy}{Little
Boy}.\textsuperscript{\hyperref[cite_note-hironaga-19]{{[}18{]}}} The
\href{/wiki/International_Space_Station}{International Space Station},
with a mass of approximately {{}450~\href{/wiki/Megagrams}{megagrams}}
and orbital velocity of
{{}7700~\href{/wiki/Metre_per_second}{m/s}},\textsuperscript{\hyperref[cite_note-iss-20]{{[}19{]}}}
has a \href{/wiki/Kinetic_energy}{kinetic energy} of roughly {{}13~TJ}.
In 2017, \href{/wiki/Hurricane_Irma}{Hurricane Irma} was estimated to
have a peak wind energy of
{{}112~TJ}.\textsuperscript{\hyperref[cite_note-21]{{[}20{]}}\hyperref[cite_note-22]{{[}21{]}}}
One \href{/wiki/Gigawatt-hour}{gigawatt-hour} of electricity is
{{}3.6~terajoules}.

{\phantomsection\label{Petajoule}{}{Petajoule}}

{{}210~petajoule} is about {{}50~\href{/wiki/Megatons}{megatons}} of
TNT, which is the amount of energy released by the
\href{/wiki/Tsar_Bomba}{Tsar Bomba}, the largest man-made explosion
ever. One \href{/wiki/Terawatt-hour}{terawatt-hour} of electricity is
{{}3.6~petajoules}.

{\phantomsection\label{Exajoule}{}{Exajoule}}

The \href{/wiki/2011_T\%C5\%8Dhoku_earthquake_and_tsunami}{2011 Tōhoku
earthquake and tsunami} in Japan had {{}1.41~EJ} of energy according to
its rating of 9.0 on the \href{/wiki/Moment_magnitude_scale}{moment
magnitude scale}. Yearly \href{/wiki/Energy_in_the_United_States}{U.S.
energy consumption} amounts to roughly {{}94~EJ}, and the world final
energy consumption was {{}439~EJ} in
2021.\textsuperscript{\hyperref[cite_note-23]{{[}22{]}}} One
\href{/wiki/Petawatt-hour}{petawatt-hour} of electricity is
{{}3.6~exajoules}.

{\phantomsection\label{Zettajoule}{}{Zettajoule}}

The zettajoule is somewhat more than the amount of energy required to
heat the \href{/wiki/Baltic_sea}{Baltic sea} by 1~°C, assuming
properties similar to \href{/wiki/Properties_of_water}{those of pure
water}.\textsuperscript{\hyperref[cite_note-Volumes_of_the_Worldux27s_Oceans-24]{{[}23{]}}}
Human annual \href{/wiki/World_energy_consumption}{world energy
consumption} is approximately {{}0.5~ZJ}. The energy to raise the
temperature of Earth\textquotesingle s atmosphere 1~°C is approximately
{{}2.2~ZJ}.

{\phantomsection\label{Yottajoule}{}{Yottajoule}}

The yottajoule is a little less than the amount of energy required to
heat the \href{/wiki/Indian_Ocean}{Indian Ocean} by 1~°C, assuming
properties similar to those of pure
water.\textsuperscript{\hyperref[cite_note-Volumes_of_the_Worldux27s_Oceans-24]{{[}23{]}}}
The thermal output of the \href{/wiki/Sun}{Sun} is approximately
{{}400~YJ} per second.

\subsection[{{{[}}\href{/w/index.php?title=Joule&action=edit&section=5}{{edit}}{{]}}}]{\texorpdfstring{\phantomsection\label{Conversions}{Conversions}{{{[}}\href{/w/index.php?title=Joule&action=edit&section=5}{{edit}}{{]}}}}{Conversions{[}edit{]}}}\label{conversionsedit}

Main article: \href{/wiki/Conversion_of_units_of_energy}{Conversion of
units of energy}

1 joule is equal to (approximately unless otherwise stated):

\begin{itemize}
\tightlist
\item
  {{}10\textsuperscript{7}~\href{/wiki/Erg}{erg}} (exactly)
\item
  {{}6.241{509}{74}{×}10\textsuperscript{18}~\href{/wiki/Electronvolt}{eV}}
\item
  {{}0.2390~\href{/wiki/Calorie}{cal}} (gram calories)
\item
  {{}2.390{×}10\textsuperscript{−4}~\href{/wiki/Calorie}{kcal}} (food
  calories)
\item
  {{}9.4782{×}10\textsuperscript{−4}~\href{/wiki/British_thermal_unit}{BTU}}
\item
  {{}0.7376~\href{/wiki/Ft\%E2\%8B\%85lb}{ft⋅lb}} (foot-pound)
\item
  {{}23.7~\href{/wiki/Foot-poundal}{ft⋅pdl}} (foot-poundal)
\item
  {{}2.7778{×}10\textsuperscript{−7}~\href{/wiki/KW\%E2\%8B\%85h}{kW⋅h}}
  (kilowatt-hour)
\item
  {{}2.7778{×}10\textsuperscript{−4}~W⋅h} (watt-hour)
\item
  {{}9.8692{×}10\textsuperscript{−3}~latm} (litre-atmosphere)
\item
  {{}11.1265{×}10\textsuperscript{−15}~g⋅\emph{c}\textsuperscript{2}}
  (by way of
  \href{/wiki/Mass\%E2\%80\%93energy_equivalence}{mass--energy
  equivalence})
\end{itemize}

Units defined exactly in terms of the joule include:

\begin{itemize}
\tightlist
\item
  1 thermochemical \href{/wiki/Calorie}{calorie} =
  4.184{~}J\textsuperscript{\hyperref[cite_note-FAO-25]{{[}24{]}}}
\item
  1 International Table calorie =
  4.1868{~}J\textsuperscript{\hyperref[cite_note-26]{{[}25{]}}}
\item
  1{~}W⋅h = 3600{~}J (or 3.6{~}kJ)
\item
  1{~}kW⋅h = {{}3.6{×}10\textsuperscript{6}~J} (or 3.6{~}MJ)
\item
  1{~}W⋅s = {{}1~J}
\item
  1{~}\href{/wiki/Ton_TNT}{ton TNT} = {{}4.184~GJ}
\item
  1{~}\href{/wiki/Foe_(unit)}{foe} = {{}10\textsuperscript{44}~J}
\end{itemize}

\subsection[{{{[}}\href{/w/index.php?title=Joule&action=edit&section=6}{{edit}}{{]}}}]{\texorpdfstring{\phantomsection\label{Newton-metre_and_torque}{Newton-metre
and
torque}{{{[}}\href{/w/index.php?title=Joule&action=edit&section=6}{{edit}}{{]}}}}{Newton-metre and torque{[}edit{]}}}\label{newton-metre-and-torqueedit}

Main article: \href{/wiki/Newton-metre}{Newton-metre}

In \href{/wiki/Mechanics}{mechanics}, the concept of
\href{/wiki/Force}{force} (in some direction) has a close analogue in
the concept of \href{/wiki/Torque}{torque} (about some angle):

\begin{longtable}[]{@{}ll@{}}
\toprule\noalign{}
\endhead
\bottomrule\noalign{}
\endlastfoot
Linear & Angular \\
Force & Torque \\
\href{/wiki/Mass}{Mass} & \href{/wiki/Moment_of_inertia}{Moment of
inertia} \\
Displacement & Angle \\
\end{longtable}

A result of this similarity is that the SI unit for torque is the
\href{/wiki/Newton-metre}{newton-metre}, which works out
\href{/wiki/Algebra}{algebraically} to have the same
\href{/wiki/Dimensional_analysis}{dimensions} as the joule, but they are
not interchangeable. The
\href{/wiki/General_Conference_on_Weights_and_Measures}{General
Conference on Weights and Measures} has given the unit of
\href{/wiki/Energy}{energy} the name \emph{joule}, but has not given the
unit of torque any special name, hence it is simply the newton-metre
(N⋅m) -- a compound name derived from its constituent
parts.\textsuperscript{\hyperref[cite_note-BIPM2-27]{{[}26{]}}} The use
of newton-metres for torque but joules for energy is helpful to avoid
misunderstandings and
miscommunication.\textsuperscript{\hyperref[cite_note-BIPM2-27]{{[}26{]}}}

The distinction may be seen also in the fact that energy is a
\href{/wiki/Scalar_(physics)}{scalar} quantity -- the
\href{/wiki/Dot_product}{dot product} of a force
\href{/wiki/Euclidean_vector}{vector} and a displacement vector. By
contrast, torque is a vector -- the \href{/wiki/Cross_product}{cross
product} of a force vector and a distance vector. Torque and energy are
related to one another by the equation

\[E = \tau\theta\,,\]

\includegraphics{https://wikimedia.org/api/rest_v1/media/math/render/svg/5c17780b00a1e8e7e1706205dd69357162e7ce29}

where \emph{E} is energy, \emph{τ} is (the
\href{/wiki/Magnitude_(mathematics)\#Euclidean_vector_space}{vector
magnitude} of) torque, and \emph{θ} is the angle swept (in
\href{/wiki/Radian}{radians}). Since plane angles are dimensionless, it
follows that torque and energy have the same dimensions.

\subsection[{{{[}}\href{/w/index.php?title=Joule&action=edit&section=7}{{edit}}{{]}}}]{\texorpdfstring{\phantomsection\label{Watt-second}{Watt-second}{{{[}}\href{/w/index.php?title=Joule&action=edit&section=7}{{edit}}{{]}}}}{Watt-second{[}edit{]}}}\label{watt-secondedit}

A \textbf{watt-second} (symbol \textbf{W~s} or \textbf{W⋅s}) is a
\href{/wiki/Derived_unit}{derived unit} of \href{/wiki/Energy}{energy}
equivalent to the
joule.\textsuperscript{\hyperref[cite_note-28]{{[}27{]}}} The
watt-second is the energy equivalent to the power of one
\href{/wiki/Watt}{watt} sustained for one \href{/wiki/Second}{second}.
While the watt-second is equivalent to the joule in both units and
meaning, there are some contexts in which the term "watt-second" is used
instead of "joule", such as in the rating of photographic
\href{/wiki/Flash_(photography)}{electronic flash units}.
\textsuperscript{\hyperref[cite_note-29]{{[}28{]}}}

\subsection[{{{[}}\href{/w/index.php?title=Joule&action=edit&section=8}{{edit}}{{]}}}]{\texorpdfstring{\phantomsection\label{Notes}{Notes}{{{[}}\href{/w/index.php?title=Joule&action=edit&section=8}{{edit}}{{]}}}}{Notes{[}edit{]}}}\label{notesedit}

\begin{enumerate}
\tightlist
\item
  \phantomsection\label{cite_note-11}{{\textbf{\hyperref[cite_ref-11]{\^{}}}}
  {This is called the \href{/wiki/Basal_metabolic_rate}{basal metabolic
  rate}. It corresponds to about 5,000~kJ (1,200~kcal) per day. The
  kilocalorie (symbol kcal) is also known as the
  \href{/wiki/Calorie}{dietary calorie}. }}
\end{enumerate}

\subsection[{{{[}}\href{/w/index.php?title=Joule&action=edit&section=9}{{edit}}{{]}}}]{\texorpdfstring{\phantomsection\label{References}{References}{{{[}}\href{/w/index.php?title=Joule&action=edit&section=9}{{edit}}{{]}}}}{References{[}edit{]}}}\label{referencesedit}

\begin{enumerate}
\tightlist
\item
  \phantomsection\label{cite_note-1}{{\textbf{\hyperref[cite_ref-1]{\^{}}}}
  {}}

  \href{/wiki/International_Bureau_of_Weights_and_Measures}{International
  Bureau of Weights and Measures} (2006),
  \href{https://www.bipm.org/documents/20126/41483022/si_brochure_8.pdf}{\emph{The
  International System of Units (SI)}} {(PDF)} (8th~ed.), p.~120,
  \href{/wiki/ISBN_(identifier)}{ISBN}~\href{/wiki/Special:BookSources/92-822-2213-6}{92-822-2213-6},
  \href{https://web.archive.org/web/20210604163219/https://www.bipm.org/documents/20126/41483022/si_brochure_8.pdf}{archived}
  {(PDF)} from the original on 2021-06-04{, retrieved {2021-12-16}}{}
\item
  \phantomsection\label{cite_note-2}{{\textbf{\hyperref[cite_ref-2]{\^{}}}}
  {\href{https://web.archive.org/web/20060413141420/http://education.yahoo.com/reference/dictionary/entry/joule}{American
  Heritage Dictionary of the English Language}, Online Edition (2009).
  Houghton Mifflin Co., hosted by
  \href{https://web.archive.org/web/20010502171832/http://education.yahoo.com/}{Yahoo!
  Education}.}}
\item
  \phantomsection\label{cite_note-3}{{\textbf{\hyperref[cite_ref-3]{\^{}}}}
  {\emph{The American Heritage Dictionary}, Second College Edition
  (1985). Boston: Houghton Mifflin Co., p. 691.}}
\item
  \phantomsection\label{cite_note-4}{{\textbf{\hyperref[cite_ref-4]{\^{}}}}
  {\emph{McGraw-Hill Dictionary of Physics}, Fifth Edition (1997).
  McGraw-Hill, Inc., p. 224.}}
\item
  \phantomsection\label{cite_note-5}{{\textbf{\hyperref[cite_ref-5]{\^{}}}}
  {\href{/wiki/Carl_Wilhelm_Siemens}{Siemens, Cal Wilhelm} (August
  1882).
  \href{http://gallica.bnf.fr/ark:/12148/bpt6k781656}{\emph{Report of
  the Fifty-Second Meeting of the British Association for the
  Advancement of Science}}. Southhampton. pp.~1--33. pp.~6--7: ``The
  unit of heat has hitherto been taken variously as the heat required to
  raise a pound of water at the freezing-point through 1°~Fahrenheit or
  Centigrade, or, again, the heat necessary to raise a kilogramme of
  water 1°~Centigrade. The inconvenience of a unit so entirely arbitrary
  is sufficiently apparent to justify the introduction of one based on
  the electro-magnetic system, viz. the heat generated in one second by
  the current of an Ampère flowing through the resistance of an Ohm. In
  absolute measure its value is 10\textsuperscript{7} C.G.S. units, and,
  assuming Joule\textquotesingle s equivalent as 42,000,000, it is the
  heat necessary to raise 0.238~grammes of water 1°~Centigrade, or,
  approximately, the'' }}

  {{1}⁄{1000}}th part of the arbitrary unit of a pound of water raised
  1°~Fahrenheit and the {{1}⁄{4000}}th of the kilogramme of water raised
  1°~Centigrade. Such a heat unit, if found acceptable, might with great
  propriety, I think, be called the Joule, after the man who has done so
  much to develop the dynamical theory of heat.{}
\item
  \phantomsection\label{cite_note-6}{{\textbf{\hyperref[cite_ref-6]{\^{}}}}
  {Pat Naughtin:
  \href{http://www.metricationmatters.com/docs/MetricationTimeline.pdf}{\emph{A
  chronological history of the modern metric system}},
  metricationmatters.com, 2009.}}
\item
  \phantomsection\label{cite_note-7}{{\textbf{\hyperref[cite_ref-7]{\^{}}}}
  {\href{https://archive.org/details/proceedingsinte01chicgoog}{\emph{Proceedings
  of the International Electrical Congress}}. New York: American
  Institute of Electrical Engineers. 1894.{}}}
\item
  \phantomsection\label{cite_note-8}{{\textbf{\hyperref[cite_ref-8]{\^{}}}}
  {\href{http://www.bipm.org/en/CIPM/db/1946/2/}{\emph{CIPM, 1946,
  Resolution 2, Definitions of electric units}}. \emph{bipm.org}.}}
\item
  \phantomsection\label{cite_note-9}{{\textbf{\hyperref[cite_ref-9]{\^{}}}}
  {\href{http://www.bipm.org/en/CGPM/db/9/3/}{\emph{9th CGPM, Resolution
  3: Triple point of water; thermodynamic scale with a single fixed
  point; unit of quantity of heat (joule).}}, \emph{bipm.org.}}}
\item
  \phantomsection\label{cite_note-10}{{\textbf{\hyperref[cite_ref-10]{\^{}}}}
  {\href{http://www.engineeringtoolbox.com/heat-units-d_664.html}{"Units
  of Heat -- BTU, Calorie and Joule"}. \emph{Engineering Toolbox}{.
  Retrieved {2021-06-14}}.{}}}
\item
  \phantomsection\label{cite_note-12}{{\textbf{\hyperref[cite_ref-12]{\^{}}}}
  {Ristinen, Robert A.; Kraushaar, Jack J. (2006).
  {\href{https://archive.org/details/energyenvironmen00rist}{\emph{Energy
  and the Environment}}} (2nd~ed.). Hoboken, NJ: John Wiley \& Sons.
  \href{/wiki/ISBN_(identifier)}{ISBN}~\href{/wiki/Special:BookSources/0-471-73989-8}{0-471-73989-8}.{}}}
\item
  \phantomsection\label{cite_note-13}{{\textbf{\hyperref[cite_ref-13]{\^{}}}}
  {\href{https://web.archive.org/web/20121213173112/https://public.web.cern.ch/public/en/Science/Glossary-en.php}{"Physics
  -- CERN"}. \emph{public.web.cern.ch}. Archived from
  \href{http://public.web.cern.ch/Public/en/Science/Glossary-en.php}{the
  original} on 2012-12-13.{}}}
\item
  \phantomsection\label{cite_note-Cal_vs_kJ-14}{{\textbf{\hyperref[cite_ref-Cal_vs_kJ_14-0]{\^{}}}}
  {\href{https://web.archive.org/web/20230515172626/https://www.coca-colacompany.com/au/news/you-say-calorie--we-say-kilojoule-who-s-right-}{"You
  Say Calorie, We Say Kilojoule: Who\textquotesingle s Right?"}{.
  Retrieved {2 May} 2017}.{}}}
\item
  \phantomsection\label{cite_note-TSI-15}{{\textbf{\hyperref[cite_ref-TSI_15-0]{\^{}}}}
  {\href{https://web.archive.org/web/20110830221302/http://www.pmodwrc.ch/pmod.php?topic=tsi\%2Fcomposite\%2FSolarConstant}{"Construction
  of a Composite Total Solar Irradiance (TSI) Time Series from 1978 to
  present"}. Archived from
  \href{http://www.pmodwrc.ch/pmod.php?topic=tsi/composite/SolarConstant}{the
  original} on 2011-08-30{. Retrieved {2005-10-05}}.{}}}
\item
  \phantomsection\label{cite_note-16}{{\textbf{\hyperref[cite_ref-16]{\^{}}}}
  {{{\(\frac{1}{2} \cdot 70\ \text{kg} \cdot \left( {10\ \text{m/s}} \right)^{2} = 3500\ \text{J}\)}\includegraphics{https://wikimedia.org/api/rest_v1/media/math/render/svg/823e7d723a3e0ff58fdc3f36f7df8a051e40f5d9}}}}
\item
  \phantomsection\label{cite_note-17}{{\textbf{\hyperref[cite_ref-17]{\^{}}}}
  {{{\(\frac{1}{2} \cdot 35\ \text{kg} \cdot \left( {35\ \text{m/s}} \right)^{2} = 21,400\ \text{J}\)}\includegraphics{https://wikimedia.org/api/rest_v1/media/math/render/svg/e9f7e1753b830392bd9771fcdeec6078d9434d5a}}}}
\item
  \phantomsection\label{cite_note-18}{{\textbf{\hyperref[cite_ref-18]{\^{}}}}
  {\href{https://www.eia.gov/energyexplained/index.cfm?page=about_energy_units}{"Energy
  Units -- Energy Explained, Your Guide To Understanding Energy --
  Energy Information Administration"}. \emph{www.eia.gov}.{}}}
\item
  \phantomsection\label{cite_note-hironaga-19}{{\textbf{\hyperref[cite_ref-hironaga_19-0]{\^{}}}}
  {Malik, John (September 1985).
  \href{https://web.archive.org/web/20091011030043/http://www.mbe.doe.gov/me70/manhattan/publications/LANLHiroshimaNagasakiYields.pdf}{"Report
  LA-8819: The yields of the Hiroshima and Nagasaki nuclear explosions"}
  {(PDF)}. \href{/wiki/Los_Alamos_National_Laboratory}{Los Alamos
  National Laboratory}. Archived from
  \href{http://www.mbe.doe.gov/me70/manhattan/publications/LANLHiroshimaNagasakiYields.pdf}{the
  original} {(PDF)} on 11 October 2009{. Retrieved {18 March} 2015}.{}}}
\item
  \phantomsection\label{cite_note-iss-20}{{\textbf{\hyperref[cite_ref-iss_20-0]{\^{}}}}
  {\href{https://web.archive.org/web/20110721012349/http://www.spaceflight.esa.int/users/downloads/factsheets/fs001_12_iss.pdf}{"International
  Space Station Final Configuration"} {(PDF)}.
  \href{/wiki/European_Space_Agency}{European Space Agency}. Archived
  from
  \href{http://www.spaceflight.esa.int/users/downloads/factsheets/fs001_12_iss.pdf}{the
  original} {(PDF)} on 21 July 2011{. Retrieved {18 March} 2015}.{}}}
\item
  \phantomsection\label{cite_note-21}{{\textbf{\hyperref[cite_ref-21]{\^{}}}}
  {Bonnie Berkowitz; Laris Karklis; Reuben Fischer-Baum; Chiqui Esteban
  (11 September 2017).
  \href{https://www.washingtonpost.com/graphics/2017/national/how-big-is-hurricane-irma/}{"Analysis
  -- How Big Is Hurricane Irma?"}. \emph{Washington Post}{. Retrieved {2
  November} 2017}.{}}}
\item
  \phantomsection\label{cite_note-22}{{\textbf{\hyperref[cite_ref-22]{\^{}}}}
  {"\href{https://www.ft.com/content/2c58ce3e-9621-11e7-b83c-9588e51488a0}{Irma
  unleashes its fury on south Florida}", \emph{Financial Times},
  accessed 10-Sept-2017 {(subscription required)}}}
\item
  \phantomsection\label{cite_note-23}{{\textbf{\hyperref[cite_ref-23]{\^{}}}}
  {\href{https://www.iea.org/reports/world-energy-outlook-2022}{World
  Energy Outlook 2022} (Report). International Energy Agency. 2022.
  p.~239{. Retrieved {7 September} 2023}.{}}}
\item
  \phantomsection\label{cite_note-Volumes_of_the_Worldux27s_Oceans-24}{{\^{}
  \hyperref[cite_ref-Volumes_of_the_Worldux27s_Oceans_24-0]{\textsuperscript{\emph{\textbf{a}}}}
  \hyperref[cite_ref-Volumes_of_the_Worldux27s_Oceans_24-1]{\textsuperscript{\emph{\textbf{b}}}}}
  {\href{https://ngdc.noaa.gov/mgg/global/etopo1_ocean_volumes.html}{"Volumes
  of the World\textquotesingle s Oceans from ETOPO1"}. \emph{noaa.gov}.
  National Oceanic and Atmospheric Administration. 19 August 2020{.
  Retrieved {8 March} 2022}.{}}}
\item
  \phantomsection\label{cite_note-FAO-25}{{\textbf{\hyperref[cite_ref-FAO_25-0]{\^{}}}}
  {\href{http://www.fao.org/docrep/meeting/009/ae906e/ae906e17.htm}{The
  adoption of joules as units of energy}, FAO/WHO Ad Hoc Committee of
  Experts on Energy and Protein, 1971. A report on the changeover from
  calories to joules in nutrition.}}
\item
  \phantomsection\label{cite_note-26}{{\textbf{\hyperref[cite_ref-26]{\^{}}}}
  {\href{/wiki/Richard_Feynman}{Feynman, Richard} (1963).
  \href{http://www.numericana.com/answer/feynman.htm}{"Physical Units"}.
  \emph{Feynman\textquotesingle s Lectures on Physics}{. Retrieved
  {2014-03-07}}.{}}}
\item
  \phantomsection\label{cite_note-BIPM2-27}{{\^{}
  \hyperref[cite_ref-BIPM2_27-0]{\textsuperscript{\emph{\textbf{a}}}}
  \hyperref[cite_ref-BIPM2_27-1]{\textsuperscript{\emph{\textbf{b}}}}}
  {\href{https://web.archive.org/web/20090628084157/http://www.bipm.org/en/si/si_brochure/chapter2/2-2/2-2-2.html}{"Units
  with special names and symbols; units that incorporate special names
  and symbols"}.
  \href{/wiki/International_Bureau_of_Weights_and_Measures}{International
  Bureau of Weights and Measures}. Archived from
  \href{http://www.bipm.org/en/si/si_brochure/chapter2/2-2/2-2-2.html}{the
  original} on 28 June 2009{. Retrieved {18 March} 2015}. ``A derived
  unit can often be expressed in different ways by combining base units
  with derived units having special names. Joule, for example, may
  formally be written newton metre, or kilogram metre squared per second
  squared. This, however, is an algebraic freedom to be governed by
  common sense physical considerations; in a given situation some forms
  may be more helpful than others. In practice, with certain quantities,
  preference is given to the use of certain special unit names, or
  combinations of unit names, to facilitate the distinction between
  different quantities having the same dimension.''{}}}
\item
  \phantomsection\label{cite_note-28}{{\textbf{\hyperref[cite_ref-28]{\^{}}}}
  {\href{/wiki/International_Bureau_of_Weights_and_Measures}{International
  Bureau of Weights and Measures} (2006),
  \href{https://www.bipm.org/documents/20126/41483022/si_brochure_8.pdf}{\emph{The
  International System of Units (SI)}} {(PDF)} (8th~ed.), pp.~39--40,
  53,
  \href{/wiki/ISBN_(identifier)}{ISBN}~\href{/wiki/Special:BookSources/92-822-2213-6}{92-822-2213-6},
  \href{https://web.archive.org/web/20210604163219/https://www.bipm.org/documents/20126/41483022/si_brochure_8.pdf}{archived}
  {(PDF)} from the original on 2021-06-04{, retrieved {2021-12-16}}{}}}
\item
  \phantomsection\label{cite_note-29}{{\textbf{\hyperref[cite_ref-29]{\^{}}}}
  {\href{http://www.imaginginfo.com/print/Studio-Photography/What-Is-A-Watt-Second/3$1043}{"What
  Is A Watt Second?"}.{}}}
\end{enumerate}

\subsection[{{{[}}\href{/w/index.php?title=Joule&action=edit&section=10}{{edit}}{{]}}}]{\texorpdfstring{\phantomsection\label{External_links}{External
links}{{{[}}\href{/w/index.php?title=Joule&action=edit&section=10}{{edit}}{{]}}}}{External links{[}edit{]}}}\label{external-linksedit}

\begin{itemize}
\tightlist
\item
  {\href{/wiki/File:Wiktionary-logo-en-v2.svg}{\includegraphics[width=0.16667in,height=0.16667in]{//upload.wikimedia.org/wikipedia/commons/thumb/9/99/Wiktionary-logo-en-v2.svg/16px-Wiktionary-logo-en-v2.svg.png}}}
  The dictionary definition of
  \href{https://en.wiktionary.org/wiki/joule}{\emph{joule}} at
  Wiktionary
\end{itemize}

\begin{longtable}[]{@{}
  >{\raggedright\arraybackslash}p{(\columnwidth - 4\tabcolsep) * \real{0.3333}}
  >{\raggedright\arraybackslash}p{(\columnwidth - 4\tabcolsep) * \real{0.3333}}
  >{\raggedright\arraybackslash}p{(\columnwidth - 4\tabcolsep) * \real{0.3333}}@{}}
\toprule\noalign{}
\endhead
\bottomrule\noalign{}
\endlastfoot
\multicolumn{3}{@{}>{\raggedright\arraybackslash}p{(\columnwidth - 4\tabcolsep) * \real{1.0000} + 4\tabcolsep}@{}}{%
\begin{minipage}[t]{\linewidth}\raggedright
\begin{itemize}
\tightlist
\item
  \href{/wiki/Template:SI_units}{v}
\item
  \href{/wiki/Template_talk:SI_units}{t}
\item
  \href{/wiki/Special:EditPage/Template:SI_units}{e}
\end{itemize}

\phantomsection\label{SI_units}
\href{/wiki/International_System_of_Units}{SI units}
\end{minipage}} \\
\href{/wiki/SI_base_unit}{Base units} &
\begin{minipage}[t]{\linewidth}\raggedright
\begin{itemize}
\tightlist
\item
  \href{/wiki/Ampere}{ampere}
\item
  \href{/wiki/Candela}{candela}
\item
  \href{/wiki/Kelvin}{kelvin}
\item
  \href{/wiki/Kilogram}{kilogram}
\item
  \href{/wiki/Metre}{metre}
\item
  \href{/wiki/Mole_(unit)}{mole}
\item
  \href{/wiki/Second}{second}
\end{itemize}
\end{minipage} &
\multirow{4}{=}{\begin{minipage}[t]{\linewidth}\raggedright
{\href{/wiki/File:International_System_Of_Units_Logo_.png}{\includegraphics[width=1.19792in,height=1.19792in]{//upload.wikimedia.org/wikipedia/commons/thumb/3/30/International_System_Of_Units_Logo_.png/115px-International_System_Of_Units_Logo_.png}}}
\end{minipage}} \\
\begin{minipage}[t]{\linewidth}\raggedright
\href{/wiki/SI_derived_unit}{Derived units\\
with special names}\strut
\end{minipage} & \begin{minipage}[t]{\linewidth}\raggedright
\begin{itemize}
\tightlist
\item
  \href{/wiki/Becquerel}{becquerel}
\item
  \href{/wiki/Coulomb}{coulomb}
\item
  \href{/wiki/Celsius}{degree Celsius}
\item
  \href{/wiki/Farad}{farad}
\item
  \href{/wiki/Gray_(unit)}{gray}
\item
  \href{/wiki/Henry_(unit)}{henry}
\item
  \href{/wiki/Hertz}{hertz}
\item
  {joule}
\item
  \href{/wiki/Katal}{katal}
\item
  \href{/wiki/Lumen_(unit)}{lumen}
\item
  \href{/wiki/Lux}{lux}
\item
  \href{/wiki/Newton_(unit)}{newton}
\item
  \href{/wiki/Newton-metre}{Nm}
\item
  \href{/wiki/Ohm}{ohm}
\item
  \href{/wiki/Pascal_(unit)}{pascal}
\item
  \href{/wiki/Radian}{radian}
\item
  \href{/wiki/Siemens_(unit)}{siemens}
\item
  \href{/wiki/Sievert}{sievert}
\item
  \href{/wiki/Steradian}{steradian}
\item
  \href{/wiki/Tesla_(unit)}{tesla}
\item
  \href{/wiki/Volt}{volt}
\item
  \href{/wiki/Watt}{watt}
\item
  \href{/wiki/Weber_(unit)}{weber}
\end{itemize}
\end{minipage} \\
\href{/wiki/Non-SI_units_mentioned_in_the_SI}{Other accepted units} &
\begin{minipage}[t]{\linewidth}\raggedright
\begin{itemize}
\tightlist
\item
  \href{/wiki/Astronomical_unit}{astronomical unit}
\item
  \href{/wiki/Dalton_(unit)}{dalton}
\item
  \href{/wiki/Day}{day}
\item
  \href{/wiki/Decibel}{decibel}
\item
  \href{/wiki/Degree_(angle)}{degree of arc}
\item
  \href{/wiki/Electronvolt}{electronvolt}
\item
  \href{/wiki/Hectare}{hectare}
\item
  \href{/wiki/Hour}{hour}
\item
  \href{/wiki/Litre}{litre}
\item
  \href{/wiki/Minute}{minute}
\item
  \href{/wiki/Minute_and_second_of_arc}{minute and second of arc}
\item
  \href{/wiki/Neper}{neper}
\item
  \href{/wiki/Tonne}{tonne}
\end{itemize}
\end{minipage} \\
See also & \begin{minipage}[t]{\linewidth}\raggedright
\begin{itemize}
\tightlist
\item
  \href{/wiki/Conversion_of_units}{Conversion of units}
\item
  \href{/wiki/Metric_prefix}{Metric prefixes}
\item
  \href{/wiki/2005\%E2\%80\%932019_definitions_of_the_SI_base_units}{2005--2019
  definition}
\item
  \href{/wiki/2019_redefinition_of_the_SI_base_units}{2019 redefinition}
\item
  \href{/wiki/System_of_measurement}{Systems of measurement}
\end{itemize}
\end{minipage} \\
\multicolumn{3}{@{}>{\raggedright\arraybackslash}p{(\columnwidth - 4\tabcolsep) * \real{1.0000} + 4\tabcolsep}@{}}{%
\begin{minipage}[t]{\linewidth}\raggedright
\begin{itemize}
\tightlist
\item
  {{\includegraphics[width=0.16667in,height=0.16667in]{//upload.wikimedia.org/wikipedia/en/thumb/9/96/Symbol_category_class.svg/16px-Symbol_category_class.svg.png}}}
  \href{/wiki/Category:SI_units}{Category}
\end{itemize}
\end{minipage}} \\
\end{longtable}

\begin{longtable}[]{@{}
  >{\raggedright\arraybackslash}p{(\columnwidth - 2\tabcolsep) * \real{0.5000}}
  >{\raggedright\arraybackslash}p{(\columnwidth - 2\tabcolsep) * \real{0.5000}}@{}}
\toprule\noalign{}
\endhead
\bottomrule\noalign{}
\endlastfoot
\multicolumn{2}{@{}>{\raggedright\arraybackslash}p{(\columnwidth - 2\tabcolsep) * \real{1.0000} + 2\tabcolsep}@{}}{%
\begin{minipage}[t]{\linewidth}\raggedright
\begin{itemize}
\tightlist
\item
  \href{/wiki/Template:Energy_footer}{v}
\item
  \href{/wiki/Template_talk:Energy_footer}{t}
\item
  \href{/wiki/Special:EditPage/Template:Energy_footer}{e}
\end{itemize}

\phantomsection\label{Energy}
\href{/wiki/Energy}{Energy}
\end{minipage}} \\
\multicolumn{2}{@{}>{\raggedright\arraybackslash}p{(\columnwidth - 2\tabcolsep) * \real{1.0000} + 2\tabcolsep}@{}}{%
\begin{minipage}[t]{\linewidth}\raggedright
\begin{itemize}
\tightlist
\item
  \href{/wiki/Outline_of_energy}{Outline}
\item
  \href{/wiki/History_of_energy}{History}
\item
  \href{/wiki/Index_of_energy_articles}{Index}
\end{itemize}
\end{minipage}} \\
\begin{minipage}[t]{\linewidth}\raggedright
Fundamental\\
concepts\strut
\end{minipage} & \begin{minipage}[t]{\linewidth}\raggedright
\begin{itemize}
\tightlist
\item
  \href{/wiki/Energy}{Energy}

  \begin{itemize}
  \tightlist
  \item
    \href{/wiki/Units_of_energy}{Units}
  \end{itemize}
\item
  \href{/wiki/Conservation_of_energy}{Conservation of energy}
\item
  \href{/wiki/Thermodynamics}{Energetics}
\item
  \href{/wiki/Energy_transformation}{Energy transformation}
\item
  \href{/wiki/Energy_condition}{Energy condition}
\item
  \href{/wiki/Energy_transition}{Energy transition}
\item
  \href{/wiki/Energy_level}{Energy level}
\item
  \href{/wiki/Energy_system}{Energy system}
\item
  \href{/wiki/Mass}{Mass}

  \begin{itemize}
  \tightlist
  \item
    \href{/wiki/Negative_mass}{Negative mass}
  \item
    \href{/wiki/Mass\%E2\%80\%93energy_equivalence}{Mass--energy
    equivalence}
  \end{itemize}
\item
  \href{/wiki/Power_(physics)}{Power}
\item
  \href{/wiki/Thermodynamics}{Thermodynamics}

  \begin{itemize}
  \tightlist
  \item
    \href{/wiki/Quantum_thermodynamics}{Quantum thermodynamics}
  \item
    \href{/wiki/Laws_of_thermodynamics}{Laws of thermodynamics}
  \item
    \href{/wiki/Thermodynamic_system}{Thermodynamic system}
  \item
    \href{/wiki/Thermodynamic_state}{Thermodynamic state}
  \item
    \href{/wiki/Thermodynamic_potential}{Thermodynamic potential}
  \item
    \href{/wiki/Thermodynamic_free_energy}{Thermodynamic free energy}
  \item
    \href{/wiki/Irreversible_process}{Irreversible process}
  \item
    \href{/wiki/Thermal_reservoir}{Thermal reservoir}
  \item
    \href{/wiki/Heat_transfer}{Heat transfer}
  \item
    \href{/wiki/Heat_capacity}{Heat capacity}
  \item
    \href{/wiki/Volume_(thermodynamics)}{Volume (thermodynamics)}
  \item
    \href{/wiki/Thermodynamic_equilibrium}{Thermodynamic equilibrium}
  \item
    \href{/wiki/Thermal_equilibrium}{Thermal equilibrium}
  \item
    \href{/wiki/Thermodynamic_temperature}{Thermodynamic temperature}
  \item
    \href{/wiki/Isolated_system}{Isolated system}
  \item
    \href{/wiki/Entropy}{Entropy}
  \item
    \href{/wiki/Free_entropy}{Free entropy}
  \item
    \href{/wiki/Entropic_force}{Entropic force}
  \item
    \href{/wiki/Negentropy}{Negentropy}
  \item
    \href{/wiki/Work_(physics)}{Work}
  \item
    \href{/wiki/Exergy}{Exergy}
  \item
    \href{/wiki/Enthalpy}{Enthalpy}
  \end{itemize}
\end{itemize}
\end{minipage} \\
Types & \begin{minipage}[t]{\linewidth}\raggedright
\begin{itemize}
\tightlist
\item
  \href{/wiki/Kinetic_energy}{Kinetic}
\item
  \href{/wiki/Internal_energy}{Internal}
\item
  \href{/wiki/Thermal_energy}{Thermal}
\item
  \href{/wiki/Potential_energy}{Potential}
\item
  \href{/wiki/Gravitational_energy}{Gravitational}
\item
  \href{/wiki/Elastic_energy}{Elastic}
\item
  \href{/wiki/Electric_potential_energy}{Electric potential energy}
\item
  \href{/wiki/Mechanical_energy}{Mechanical}
\item
  \href{/wiki/Interatomic_potential}{Interatomic potential}
\item
  \href{/wiki/Quantum_potential}{Quantum potential}
\item
  \href{/wiki/Electrical_energy}{Electrical}
\item
  \href{/wiki/Magnetic_energy}{Magnetic}
\item
  \href{/wiki/Ionization_energy}{Ionization}
\item
  \href{/wiki/Radiant_energy}{Radiant}
\item
  \href{/wiki/Binding_energy}{Binding}
\item
  \href{/wiki/Nuclear_binding_energy}{Nuclear binding energy}
\item
  \href{/wiki/Gravitational_binding_energy}{Gravitational binding
  energy}
\item
  \href{/wiki/Quantum_chromodynamics_binding_energy}{Quantum
  chromodynamics binding energy}
\item
  \href{/wiki/Dark_energy}{Dark}
\item
  \href{/wiki/Quintessence_(physics)}{Quintessence}
\item
  \href{/wiki/Phantom_energy}{Phantom}
\item
  \href{/wiki/Negative_energy}{Negative}
\item
  \href{/wiki/Chemical_energy}{Chemical}
\item
  \href{/wiki/Rest_energy}{Rest}
\item
  \href{/wiki/Sound_energy}{Sound energy}
\item
  \href{/wiki/Surface_energy}{Surface energy}
\item
  \href{/wiki/Vacuum_energy}{Vacuum energy}
\item
  \href{/wiki/Zero-point_energy}{Zero-point energy}
\item
  \href{/wiki/Quantum_potential}{Quantum potential}
\item
  \href{/wiki/Quantum_fluctuation}{Quantum fluctuation}
\end{itemize}
\end{minipage} \\
\href{/wiki/Energy_carrier}{Energy carriers} &
\begin{minipage}[t]{\linewidth}\raggedright
\begin{itemize}
\tightlist
\item
  \href{/wiki/Radiation}{Radiation}
\item
  \href{/wiki/Enthalpy}{Enthalpy}
\item
  \href{/wiki/Mechanical_wave}{Mechanical wave}
\item
  \href{/wiki/Sound_wave}{Sound wave}
\item
  \href{/wiki/Fuel}{Fuel}

  \begin{itemize}
  \tightlist
  \item
    \href{/wiki/Fossil_fuel}{Fossil}
  \item
    \href{/wiki/Fuel_oil}{Oil}
  \end{itemize}
\item
  \href{/wiki/Hydrogen}{Hydrogen}

  \begin{itemize}
  \tightlist
  \item
    \href{/wiki/Hydrogen_fuel}{Hydrogen fuel}
  \end{itemize}
\item
  \href{/wiki/Heat}{Heat}

  \begin{itemize}
  \tightlist
  \item
    \href{/wiki/Latent_heat}{Latent heat}
  \end{itemize}
\item
  \href{/wiki/Work_(physics)}{Work}
\item
  \href{/wiki/Electricity}{Electricity}
\item
  \href{/wiki/Electric_battery}{Battery}
\item
  \href{/wiki/Capacitor}{Capacitor}
\end{itemize}
\end{minipage} \\
\href{/wiki/Primary_energy}{Primary energy} &
\begin{minipage}[t]{\linewidth}\raggedright
\begin{itemize}
\tightlist
\item
  \href{/wiki/Fossil_fuel}{Fossil fuel}

  \begin{itemize}
  \tightlist
  \item
    \href{/wiki/Coal}{Coal}
  \item
    \href{/wiki/Petroleum}{Petroleum}
  \item
    \href{/wiki/Natural_gas}{Natural gas}
  \end{itemize}
\item
  \href{/wiki/Nuclear_fuel}{Nuclear fuel}

  \begin{itemize}
  \tightlist
  \item
    \href{/wiki/Natural_uranium}{Natural uranium}
  \end{itemize}
\item
  \href{/wiki/Radiant_energy}{Radiant energy}
\item
  \href{/wiki/Solar_energy}{Solar}
\item
  \href{/wiki/Wind_energy}{Wind}
\item
  \href{/wiki/Hydropower}{Hydropower}
\item
  \href{/wiki/Marine_energy}{Marine energy}
\item
  \href{/wiki/Geothermal_energy}{Geothermal}
\item
  \href{/wiki/Bioenergy}{Bioenergy}
\item
  \href{/wiki/Gravitational_energy}{Gravitational energy}
\end{itemize}
\end{minipage} \\
\begin{minipage}[t]{\linewidth}\raggedright
\href{/wiki/Energy_system}{Energy system}\\
components\strut
\end{minipage} & \begin{minipage}[t]{\linewidth}\raggedright
\begin{itemize}
\tightlist
\item
  \href{/wiki/Energy_engineering}{Energy engineering}
\item
  \href{/wiki/Oil_refinery}{Oil refinery}
\item
  \href{/wiki/Electricity_delivery}{Electricity delivery}
\item
  \href{/wiki/Electric_power}{Electric power}
\item
  \href{/wiki/Fossil_fuel_power_station}{Fossil fuel power station}

  \begin{itemize}
  \tightlist
  \item
    \href{/wiki/Cogeneration}{Cogeneration}
  \item
    \href{/wiki/Integrated_gasification_combined_cycle}{Integrated
    gasification combined cycle}
  \end{itemize}
\item
  \href{/wiki/Nuclear_power}{Nuclear power}

  \begin{itemize}
  \tightlist
  \item
    \href{/wiki/Nuclear_power_plant}{Nuclear power plant}
  \item
    \href{/wiki/Radioisotope_thermoelectric_generator}{Radioisotope
    thermoelectric generator}
  \end{itemize}
\item
  \href{/wiki/Solar_power}{Solar power}

  \begin{itemize}
  \tightlist
  \item
    \href{/wiki/Photovoltaic_system}{Photovoltaic system}
  \item
    \href{/wiki/Concentrated_solar_power}{Concentrated solar power}
  \end{itemize}
\item
  \href{/wiki/Solar_thermal_energy}{Solar thermal energy}

  \begin{itemize}
  \tightlist
  \item
    \href{/wiki/Solar_power_tower}{Solar power tower}
  \item
    \href{/wiki/Solar_furnace}{Solar furnace}
  \end{itemize}
\item
  \href{/wiki/Wind_power}{Wind power}

  \begin{itemize}
  \tightlist
  \item
    \href{/wiki/Wind_farm}{Wind farm}
  \item
    \href{/wiki/Airborne_wind_energy}{Airborne wind energy}
  \end{itemize}
\item
  \href{/wiki/Hydropower}{Hydropower}

  \begin{itemize}
  \tightlist
  \item
    \href{/wiki/Hydroelectricity}{Hydroelectricity}
  \item
    \href{/wiki/Wave_farm}{Wave farm}
  \item
    \href{/wiki/Tidal_power}{Tidal power}
  \end{itemize}
\item
  \href{/wiki/Geothermal_power}{Geothermal power}
\item
  \href{/wiki/Biomass}{Biomass}
\end{itemize}
\end{minipage} \\
\begin{minipage}[t]{\linewidth}\raggedright
Use and\\
\href{/wiki/Energy_supply}{supply}\strut
\end{minipage} & \begin{minipage}[t]{\linewidth}\raggedright
\begin{itemize}
\tightlist
\item
  \href{/wiki/Energy_consumption}{Energy consumption}
\item
  \href{/wiki/Energy_storage}{Energy storage}
\item
  \href{/wiki/World_energy_consumption}{World energy consumption}
\item
  \href{/wiki/Energy_security}{Energy security}
\item
  \href{/wiki/Energy_conservation}{Energy conservation}
\item
  \href{/wiki/Efficient_energy_use}{Efficient energy use}

  \begin{itemize}
  \tightlist
  \item
    \href{/wiki/Energy_efficiency_in_transport}{Transport}
  \item
    \href{/wiki/Energy_efficiency_in_agriculture}{Agriculture}
  \end{itemize}
\item
  \href{/wiki/Renewable_energy}{Renewable energy}
\item
  \href{/wiki/Sustainable_energy}{Sustainable energy}
\item
  \href{/wiki/Energy_policy}{Energy policy}

  \begin{itemize}
  \tightlist
  \item
    \href{/wiki/Energy_development}{Energy development}
  \end{itemize}
\item
  \href{/wiki/Worldwide_energy_supply}{Worldwide energy supply}
\item
  \href{/wiki/Energy_in_South_America}{South America}
\item
  \href{/wiki/Energy_in_the_United_States}{United States}
\item
  \href{/wiki/Energy_in_Mexico}{Mexico}
\item
  \href{/wiki/Energy_policy_of_Canada}{Canada}
\item
  \href{/wiki/Energy_in_Europe}{Europe}
\item
  \href{/wiki/Category:Energy_in_Asia}{Asia}
\item
  \href{/wiki/Energy_in_Africa}{Africa}
\item
  \href{/wiki/Energy_in_Australia}{Australia}
\end{itemize}
\end{minipage} \\
Misc. & \begin{minipage}[t]{\linewidth}\raggedright
\begin{itemize}
\tightlist
\item
  \href{/wiki/Jevons_paradox}{Jevons paradox}
\item
  \href{/wiki/Carbon_footprint}{Carbon footprint}
\end{itemize}
\end{minipage} \\
\multicolumn{2}{@{}>{\raggedright\arraybackslash}p{(\columnwidth - 2\tabcolsep) * \real{1.0000} + 2\tabcolsep}@{}}{%
\begin{minipage}[t]{\linewidth}\raggedright
\begin{itemize}
\tightlist
\item
  {{\includegraphics[width=0.16667in,height=0.16667in]{//upload.wikimedia.org/wikipedia/en/thumb/9/96/Symbol_category_class.svg/16px-Symbol_category_class.svg.png}}}
  \textbf{\href{/wiki/Category:Energy}{Category}}
\item
  {{\includegraphics[width=0.125in,height=0.16667in]{//upload.wikimedia.org/wikipedia/en/thumb/4/4a/Commons-logo.svg/12px-Commons-logo.svg.png}}}
  \textbf{\href{https://commons.wikimedia.org/wiki/Category:Energy}{Commons}}
\item
  {\href{/wiki/File:Symbol_portal_class.svg}{\includegraphics[width=0.16667in,height=0.16667in]{//upload.wikimedia.org/wikipedia/en/thumb/e/e2/Symbol_portal_class.svg/16px-Symbol_portal_class.svg.png}}}
  \textbf{\href{/wiki/Portal:Energy}{Portal}}
\item
  {{\includegraphics[width=0.16667in,height=0.16667in]{//upload.wikimedia.org/wikipedia/commons/thumb/3/37/People_icon.svg/16px-People_icon.svg.png}}}
  \textbf{\href{/wiki/Wikipedia:WikiProject_Energy}{WikiProject}}
\end{itemize}
\end{minipage}} \\
\end{longtable}

\end{otherlanguage}
\end{LTR}

\includegraphics[width=0.01042in,height=0.01042in]{https://login.wikimedia.org/wiki/Special:CentralAutoLogin/start?type=1x1}

Retrieved from
"\url{https://en.wikipedia.org/w/index.php?title=Joule&oldid=1187127150}"

\phantomsection\label{catlinks}
\phantomsection\label{mw-normal-catlinks}
\href{/wiki/Help:Category}{Categories}:

\begin{itemize}
\tightlist
\item
  \href{/wiki/Category:James_Prescott_Joule}{James Prescott Joule}
\item
  \href{/wiki/Category:SI_derived_units}{SI derived units}
\item
  \href{/wiki/Category:Units_of_energy}{Units of energy}
\end{itemize}

\phantomsection\label{mw-hidden-catlinks}
Hidden categories:

\begin{itemize}
\tightlist
\item
  \href{/wiki/Category:Pages_containing_links_to_subscription-only_content}{Pages
  containing links to subscription-only content}
\item
  \href{/wiki/Category:Articles_with_short_description}{Articles with
  short description}
\item
  \href{/wiki/Category:Short_description_is_different_from_Wikidata}{Short
  description is different from Wikidata}
\item
  \href{/wiki/Category:Use_British_English_from_February_2022}{Use
  British English from February 2022}
\end{itemize}

\begin{itemize}
\tightlist
\item
  \phantomsection\label{footer-info-lastmod}{This page was last edited
  on 27 November 2023, at 17:09{~(UTC)}.}
\item
  \phantomsection\label{footer-info-copyright}{Text is available under
  the
  \href{//en.wikipedia.org/wiki/Wikipedia:Text_of_the_Creative_Commons_Attribution-ShareAlike_4.0_International_License}{Creative
  Commons Attribution-ShareAlike License
  4.0}\href{//en.wikipedia.org/wiki/Wikipedia:Text_of_the_Creative_Commons_Attribution-ShareAlike_4.0_International_License}{};
  additional terms may apply. By using this site, you agree to the
  \href{//foundation.wikimedia.org/wiki/Terms_of_Use}{Terms of Use} and
  \href{//foundation.wikimedia.org/wiki/Privacy_policy}{Privacy Policy}.
  Wikipedia® is a registered trademark of the
  \href{//www.wikimediafoundation.org/}{Wikimedia Foundation, Inc.}, a
  non-profit organization.}
\end{itemize}

\begin{itemize}
\tightlist
\item
  \phantomsection\label{footer-places-privacy}{\href{https://foundation.wikimedia.org/wiki/Special:MyLanguage/Policy:Privacy_policy}{Privacy
  policy}}
\item
  \phantomsection\label{footer-places-about}{\href{/wiki/Wikipedia:About}{About
  Wikipedia}}
\item
  \phantomsection\label{footer-places-disclaimers}{\href{/wiki/Wikipedia:General_disclaimer}{Disclaimers}}
\item
  \phantomsection\label{footer-places-contact}{\href{//en.wikipedia.org/wiki/Wikipedia:Contact_us}{Contact
  Wikipedia}}
\item
  \phantomsection\label{footer-places-wm-codeofconduct}{\href{https://foundation.wikimedia.org/wiki/Special:MyLanguage/Policy:Universal_Code_of_Conduct}{Code
  of Conduct}}
\item
  \phantomsection\label{footer-places-developers}{\href{https://developer.wikimedia.org}{Developers}}
\item
  \phantomsection\label{footer-places-statslink}{\href{https://stats.wikimedia.org/\#/en.wikipedia.org}{Statistics}}
\item
  \phantomsection\label{footer-places-cookiestatement}{\href{https://foundation.wikimedia.org/wiki/Special:MyLanguage/Policy:Cookie_statement}{Cookie
  statement}}
\item
  \phantomsection\label{footer-places-mobileview}{\href{//en.m.wikipedia.org/w/index.php?title=Joule&mobileaction=toggle_view_mobile}{Mobile
  view}}
\end{itemize}

\begin{itemize}
\tightlist
\item
  \phantomsection\label{footer-copyrightico}{\href{https://wikimediafoundation.org/}{\includegraphics[width=0.91667in,height=0.32292in]{/static/images/footer/wikimedia-button.png}}}
\item
  \phantomsection\label{footer-poweredbyico}{\href{https://www.mediawiki.org/}{\includegraphics[width=0.91667in,height=0.32292in]{/static/images/footer/poweredby_mediawiki_88x31.png}}}
\end{itemize}

\phantomsection\label{p-dock-bottom}
\begin{itemize}
\tightlist
\item
  {} {Toggle limited content width}
\end{itemize}

\end{document}
